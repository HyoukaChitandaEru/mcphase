\section{Theory for  program {\prg bfk}\index{bfk} - Inelastic neutron-scattering from RE ions in a crystal field
including damping effects due to the exchange interaction with conduction
electrons}\label{bfktheory}

\medskip
This is an extension  of the theory published by Klaus W. Becker, Peter Fulde and
Joachim Keller in Z. Physik B 28,9-18, 1977 
"Line width of crystal-field excitations in metallic rare-earth systems"
and an  introduction to the computer program for the calculation of  the neutron 
scattering cross section. The computer program {\prg bfk} is written by J. Keller,
University of Regensburg.

\noindent
Here we present a brief outline of the theoretical concepts to calculate the
dynamical susceptibility of the Re ions and the scattering cross section.  

The neutron-scattering cross section is related to the dynamic susceptibility
of the  RE ions 
$$
\chi_{\alpha\beta}(t)={i\over \hbar} \Theta(t)\langle [J^\dagger_\alpha(t), J_\beta(0)]\rangle 
$$
whose Fourier-Laplace transform
$$
\chi_{\alpha,\beta}(z)=\int_{-\infty}^{+\infty} dt e^{izt}\chi_{\alpha\beta}(t), \quad z=\omega
+i\delta 
$$
determines the inelastic neutron scattering crossection   
(Stephen W. Lovesey; "Theory of neutron 
scattering from condensed matter"
Vol 2, equ. 11,144).
$$
{d^2\sigma \over d\Omega d E'}=  {k' \over
k}({r_0\over 2}g_J F(Q))^2{1\over \pi }
\sum_{\alpha\beta}(\delta_{\alpha\beta} 
- \tilde Q_\alpha \tilde Q_\beta)
{ \chi{"}_{\alpha,\beta}(\omega)\over 1-e^{-\beta \hbar \omega}} 
$$
Here $k$ and
$ k'$ denote the  wave number of the neutron before and after the
scattering. $\vec Q = \vec k - \vec k'$ is the scattering wave vector,
$\tilde Q = \vec Q/\vert\vec Q\vert$. 
$r_0= -0.54 \cdot 10^{-12}$ cm is the basic scattering length, $g_J$ is the Land\'e factor, 
$F(Q)$ the atomic form factor of the
rare earth ion.
  
\bigskip
\noindent
Formal evaluation of the dynamic and static susceptiblity.

The dynamic spin-susceptibilities are correlation functions of the form
$$
\chi_{i,k}(t)=i \Theta(t) \langle [A_i^\dagger(t),A_k(0)]\rangle
$$ 
where $A(t)$ is a Heisenberg operator
$$
A(t)= \exp(iHt)A\exp(-iHt)
$$
Introducing a Liouville operator (acting on operators of dynamical
variables) by ${\cal L}A = [H,A]$  the Heisenberg operator can also be
written formally as  
$$
A(t)= \exp(i{\cal L}t) A
$$
With help of this definition the dynamical susceptibility $\chi_{i,k}$ of 
two variables $A_i, A_k$ can be written as
$$
\chi_{i,k}(t)=i \Theta(t) \langle [A_i^\dagger(t),A_k(0)]\rangle
$$ 
and their Laplace transform
$$
\chi_{i,k}(z)=i\int_0^\infty dt e^{izt} \langle [A_i^\dagger(t),A_k(0)]\rangle
$$ 
With help of the Liouvillian these quantities can be written as
$$
\chi_{i,k}(t)=i \Theta(t) \langle [A_i^\dagger,A_k\exp^{-i{\cal L}t}\rangle
$$
and their Laplace transform
$$
\chi_{i,k}(z)= -\langle [A_i^\dagger,{1\over {z-\cal
L}}A_k(0)]\rangle
$$ 

The static isothermal susceptibilities can also formally be calculated with help of
the Liouvillian.   
$$
\chi_{i,k}(0) = \int_0^\beta d \lambda \langle e^{\lambda H}
A_i^\dagger e^{-\lambda H}  A_k\rangle
 = \int_0^\beta d \lambda \langle (e^{\lambda {\cal L}}
A_i^\dagger)   A_k\rangle
$$


The static susceptibilities are used to define a scalar product between  the
dynamical variables:
$$
(A_i \vert A_k) =  {1\over \beta }\int_0^\beta d\lambda \langle
(e^{\lambda{\cal L}}A_i^\dagger)A_k\rangle ={1\over \beta} \chi_{ik}(0)
$$
It fulfills the axioms of a scalar product and  furthermore it has the important property
$$
({\cal L}A_i\vert A_k)=(A_i\vert {\cal L}A_k)={1\over \beta}\langle
[A_i^\dagger,A_k]\rangle
$$
With help of this relation the dynamical susceptibility can be
expressed as
$$
\chi_{i,k}(z)= -\beta (A_i\vert {{\cal L}\over {z-\cal L}} A_k)
$$
and finally as 
$$
\chi_{ik}(z)=\chi_{ik}(0)-z\beta (A_i\vert {1\over {z-\cal L}}A_k)  
$$
The second term is the so-called relaxation function 
$$
\Phi_{ik}(z)=(A_i \vert {1 \over {z-\cal L}}A_k)
$$

\bigskip




\noindent
The model:



We calculate the spin susceptibility of a RE ion in the presence of exchange
interaction with conduction electrons. The system is described by the
Hamiltonian
$$
H=H_{cf}+H_{el}+H_{el,cf} 
$$
The first part is the cf-Hamiltonian of the spin-system: 
$$ 
H_{cf}= \sum_n E_n K_{nn}, \quad K_{nm}= \vert n\rangle \langle m\vert
$$  
written in terms of the crystal field eigenstates $\vert n\rangle$. 
The second part is the Hamiltonian of the conduction electrons
$$
H_{el}=\sum_{k\alpha}\epsilon_kc^\dagger_{k\alpha}c_{k\alpha}
$$
The third part is the 
 interaction between local moments and  the conduction electrons
$$
H_{el,cf}= - J_{ex}\vec J \cdot \vec \sigma, \quad \vec \sigma = \sum_{k\alpha
Q\beta}\vec \sigma_{\alpha\beta}c^\dagger_{k\alpha}c_{k+Q\beta}, \quad \vec
J=\sum_{n,m}\vec J_{n,m}K_{nm}.
$$
We assume, that the energies $E_n$ and the eigenstates $\vert n\rangle$
expressed by angular momentum eigenstates are known.

\bigskip
\noindent
Definition of dynamical variables

In our case we use as dynamical variable the standard-basis operators
$$
A_\mu= K_{\mu}
$$
 describing a  transition $\mu= [nm]$ between CEF levels $m$ and $n$. 
In the absence of the interaction with conduction electrons
$$
{\cal L}A_\mu = (E_n-E_m)A_\mu
$$
In order to
get the spin suceptibility we have to multiply the final expressions by the
spin-matrixelements:
$$ 
\chi_{\alpha \beta}   
$$

The idea of the projection formalism to calculate the dynamical
susceptibility of a variable $A$ is to project this variable  onto a closed
set of dynamical variables $A_i$ and to solve approximately the coupled
equations between these variables. For this purpose a projector is defined
by
$$
{\cal P} A= \sum_{\nu \mu}A_\nu P^{-1}_{\nu \mu}(A_\mu\vert A) \quad
P_{\nu\mu}=(A_\nu\vert A_\mu)
$$ 
where $ P^{-1}_{\nu \mu}=[P^{-1}]_{\nu \mu}$ is the ${\nu\mu}$-component of the inverse matrix of
$P$. 

For the  resolvent operator of the relaxation function
$$
{\cal F}(z)= {1\over {z-\cal L}}, \quad ({z-\cal L}){\cal F}(z)=1 
$$
one obtains the exact equation
$$
({\cal P}(z-{\cal P}{\cal L}{\cal P} - {\cal P} {\cal M}(z) {\cal P}){\cal P} {\cal
F}(z) {\cal P}= {\cal P}
$$
with the memory function
$$
{\cal M}(z)={\cal PLQ}{1\over z-{\cal QLQ} }{\cal QLP}
$$ 
where ${\cal Q}=1-{\cal P}$. In components
$$
\Phi_{\nu\mu}(z)= (A_\nu\vert {1\over z-{\cal L}} A_\mu)
$$
$$
\sum_\lambda \Bigl(z\delta_{\nu\lambda}-\sum_\kappa\bigl[L_{\nu\kappa}+
M_{\nu\kappa}(z)\bigr]P^{-1}_{\kappa\lambda}\Bigr)\Phi_{\lambda\mu}(z)
=P_{\nu \mu}
$$
with
$$
L_{\nu\mu}=(A_\nu\vert {\cal L}A_\mu)
$$
and the memory function 
$$
M_{\nu\mu}(z)=(A_\nu\vert{\cal M}(z)A_\mu)
$$

Now we apply the formalism to the coupled spin-electron system and restrict
ourselves to the lowest order contributions of the spin electron
interaction. As dynamical variables we choose a decomposition of the  
original spin-variable:
$$
J^\alpha=\sum_{n_1,n_2}J^\alpha_{n_2,n_1}K_{n_2,n_1}=\sum_\nu J^\alpha_\nu A_\nu, 
\quad 
A_\nu= K_{n_1n_2}   
$$
where $\nu$ denotes a transition $n_2 \gets n_1$ performed with the
standard-basis operator $\vert n_2\rangle\langle n_1\vert$. 

In lowest (zeroth) order in the el-cf interaction  
$$
{\cal L}A_\nu = (E_{n_2}-E_{n_1)}A_\nu
$$
and  the scalar product is diagonal in lowest order in the transition
operators,
$$
P_{\nu\mu}=(A_\nu\vert A_\mu)\simeq\delta_{\nu
\mu}P_\nu, \quad P_\nu=(A_\nu\vert
A_\nu)={p(n_1)-p(n_2)\over \beta (E_{n2}-E_{n_1})}
$$
where $p(n)=\exp(-\beta E_n)/Z$ is the thermal occupation number. For the
frequency term we then get 
$$
L_{\nu\mu}=\delta_{\nu\mu}(A_\nu\vert A_\nu) (E_{n_2}-E_{n_1} )
+O(J_{ex}^2)
$$
Neglecting the second-order energy corrections in the following we obtain
the equation for the relaxation function 
$$
\Phi_{\nu\mu}(z)=\bigl[\Omega^{-1}\bigr]_{\nu\mu}(z)
P_\mu, \quad
\Omega_{\nu\mu}(z)=(z-E_\nu)\delta_{\nu\mu} -
M_{\nu\mu}(z)[P^{-1}]_\mu, \quad
E_\nu = E_{n_2}- E_{n_1}
$$
and it remains to calculate the memoryfunction containing the relaxation
processes. 

In lowest order in the electron-spin interaction ${\cal QL}A_\nu$ can be replaced by ${\cal L}_{el,cf}A_\nu$.
Then we get for the memory function
$$
M_{\nu \mu}(z)= ({\cal L}_{el,cf}A_\nu \vert{1\over z- {\cal
L}_0}{\cal L}_{el,cf}A_\mu)=
M_{n_2n_1,m_2m_1}(z)
$$
with
$$
M_{n_2n_1,m_2m_1}(z)=({\cal L}_{el,cf} K_{n_2n_1}\vert{1\over z-{\cal
L}_0}{\cal L}_{el,cf} K_{m_2m_1})
$$

Now
$$
{\cal L}_{el,cf} K_{n_2n_1} = J_{ex}\sum_t \vec \sigma(\vec  J_{n_1t}
K_{n_2t} - \vec
J_{tn_2}K_{tn_1})
$$
with
$$
\vec \sigma = \sum_{k\alpha, k+Q\beta}\vec \sigma_{\alpha\beta}
c^\dagger_{k\alpha}c_{k+Q\beta}
$$
With help of the symmetry properties 
$$
( \sigma^i K_{nm}\vert {1\over z- {\cal L}_0}\sigma^j K_{n'm'})=
\delta_{ij}\delta_{nn'}\delta_{mm'}G_{nm}(z)
$$
with
$$
G_{nm}(z)=( \sigma^i K_{nm}\vert {1\over  {z-\cal L}_0}\sigma^i K_{nm})
$$
we obtain
\begin{eqnarray}
M_{n_2n_1,m_2m_1}(z)=J_{ex}^2\sum_i\Bigl[&
\delta_{n_2m_2}\sum_tJ^i_{m_1t}J^i_{tn_1}G_{n_2t}
+ \delta_{n_1m_1}\sum_tJ^i_{
n_2t}J^i_{tm_2}G_{tn_1}\nonumber \\
&-J^i_{m_1n_1}J^i_{n_2m_2}G_{n_2m_1}
-J^i_{n_2m_2}J^i_{m_1n_1}G_{m_2n_1}\Bigr]\nonumber
\end{eqnarray}


In order to calculate the relaxation functions $G_{n,m}(z)$
we use the general relation between relaxation function and dynamic
susceptibility
$$
\chi(z)= \chi(0)-\beta z \Phi(z)
$$
and calculate instead the corresponding susceptibility (using tr
$\sigma^i\sigma^i)=2$):
\begin{eqnarray}
G_{nm}(z)&= {2\over \beta \omega}\sum_{k,k+Q} \langle \Bigl[K_{mn}
 c^\dagger_{k+Q}c_{k},(z - E_n +
E_m  -\epsilon_k+\epsilon_{k+Q})^{-1} K_{nm}
c^\dagger_{k}c_{k+Q}\Bigr]\rangle \nonumber\\
&=  {2\over \beta \omega
}\sum_{k,Q}(f_{k+Q}(1-f_{k})p_m-f_{k}(1-f_{k+Q})p_n)(
z-E_n+E_m-\epsilon_{k}+\epsilon_{k+Q})^{-1}\nonumber
\end{eqnarray}
We are interested in the imaginary part describing the relaxation processes:
$$
Im G_{nm}(\omega+i\delta)= - {2\pi \over \beta \omega
}\sum_{k,Q}\Bigl(f_{k+Q}(1-f_{k})p_m-f_{k}(1-f_{k+Q})p_n \Bigr)
 \delta(\omega -E_n+E_m-\epsilon_{k}+\epsilon_{k+Q})
$$
Writing $ \rho=\omega - \omega_{nm}$ and $\omega_{nm}=E_n-E_m$  we obtain
$$
Im G_{nm}(\omega+i\delta)= - {2\pi N^2(0)\over \beta \omega} \int d\epsilon  
 (f(\epsilon)(1-f(\epsilon+\rho) p_m
- f(\epsilon+\rho)(1-f(\epsilon))p_n)
$$
For the integrals we get
\begin{eqnarray}
\int d\epsilon f(\epsilon)(1-f(\epsilon+\rho)=&\int d\epsilon
\exp(\beta(\epsilon+\rho))/
(1+\exp(\beta\epsilon))(1+\exp(\beta(\epsilon+\rho))\nonumber\\
=& 
(\omega-\omega_{nm})\exp(\beta(\omega-\omega_{nm}))/
(-1+\exp(\beta(\omega-\omega_{nm})\nonumber
\end{eqnarray}
\begin{eqnarray}
\int d\epsilon f(\epsilon+\rho )(1-f(\epsilon)=&\int d\epsilon
\exp(\beta(\epsilon)/
(1+\exp(\beta\epsilon))(1+\exp(\beta(\epsilon+\rho))\nonumber\\
=& 
(\omega-\omega_{nm})/
(-1+\exp(\beta(\omega-\omega_{nm}))\nonumber
\end{eqnarray}
This makes
$$
Im G_{nm}=-{2\pi N^2(0)\over \beta \omega}(\omega -\omega_{nm}) {1-\exp(-\beta
\omega)\over
1-\exp[(\omega_{nm}-\omega)\beta]}p_m
$$
which has to be used to calculate the imaginary part of the memory function.
Writing 
$$
F_{nm}(\omega )= {1\over \beta \omega}(\omega -\omega_{nm}) {1-\exp(-\beta
\omega)\over
1-\exp[(\omega_{nm}-\omega)\beta]}p_m
$$
which also be written in symmetrized form as
$$
F_{nm}(\omega )= {\sqrt{p_np_m}\over \beta}{(\omega -\omega_{nm})\over
\omega} {\exp(\beta \omega)/2) - \exp(-\beta \omega)/2)\over
\exp(\beta (\omega-\omega_{nm})/2) - \exp(-\beta (\omega -\omega_{nm})/2)}
$$
we obtain with $g=J_{ex}N(0)$
\begin{eqnarray}
M_{n_2n_1,m_2m_1}(\omega) =- i 2\pi g^2
\sum_i\Bigl[&
\delta_{n_2m_2}\sum_tJ^i_{m_1t}J^i_{tn_1}F_{n_2t}
+ \delta_{n_1m_1}\sum_tJ^i_{
n_2t}J^i_{tm_2}F_{tn_1}\nonumber\\
&-J^i_{m_1n_1}J^i_{n_2m_2}F_{n_2m_1}
-J^i_{n_2m_2}J^i_{m_1n_1}F_{m_2n_1}\Bigr]
\nonumber
\end{eqnarray}
from which we get the memory function matrix in the space of dynamical variables
$$
M_{\nu \mu}(\omega)= M_{n_2n_1,m_2m_1}(\omega)
$$

\bigskip
\noindent


Summary:
For the neutron scattering cross section we need the function 
$Im \chi^{\alpha\beta}(\omega+i\delta)/(1-\exp(-\beta\omega)$, where
$\chi^{\alpha\beta}(z)$ is the frequency dependent part of the dynamic
susceptibility $\chi^{\alpha\beta}(z)$ for spin components  $J^\alpha$,$J^\beta$, which is
related to the corresponding relaxation function $\Phi^{\alpha,\beta}$ by
$$
\chi^{\alpha\beta}(z) = \chi^{\alpha\beta}(0) - \beta z \Phi^{\alpha\beta}(z)  
$$
For the full dynamical susceptibility we need the static suseptibility
$ \chi^{\alpha\beta}(0) $
which  in lowest order in the exchange interaction is given by 
$$
\chi^{\alpha\beta}(0)  = \sum_\nu  (J^\alpha_\nu)^\dagger \beta P_\nu J^\beta_\nu
$$
The above relaxation function is calculated with help of the Mori-Zwanzig
projection formalism by 
$$
\Phi^{\alpha\beta}(z)=\sum_{\mu\nu}
(J^\alpha_\nu)^*\Phi_{\nu\mu}(z)J^\beta_\mu
$$
where $\nu$ denotes a transition from $n_1$ to $n_2$ between crystal field
levels of the magnetic ion. The partial relaxation functions are obtained by 
solving the matrix
equation
$$
\Phi_{\nu\mu}(z)= [\Omega^{-1}]_{\nu\mu}P_\mu      
$$
with
$$
\Omega_{\nu\mu}(z)= (z-\omega_\nu)\delta_{\nu\mu}  -M_{\nu\mu}(z)/P_\mu
$$
where $\omega_\nu =E_{n_2}- E_{n_1}$ is the energy difference of cf-levels.

Only terms in  lowest  order in the el-ion interaction are kept. We neglect
frequency shifts due to the electron-ion interaction. 
Then the  memory function is purely
imaginary (with a negative  sign).

Note that compared to our paper BFK, Z.Physik B28, 9-18, 1977 we have used here 
a different sign-convention.

For numerical reasons it is more convenient to calculate the relaxation
function in the following way:
$$
\Phi_{\nu\mu}(z)= P_\nu[\bar\Omega^{-1}]_{\nu\mu}P_\mu      
$$
with
$$
\bar \Omega_{\nu\mu}(z)= P_\nu(z-\omega_\nu)\delta_{\nu\mu}  - M_{\nu\mu}(z)
$$

From the relaxation function we get for the dynamic scattering cross section 
$$
{d^2\sigma \over d\Omega d E'} =  {k' \over
k}S(\vec Q,\omega)
$$
with
$$
S(\vec Q, \omega)=({r_0\over 2}g_J F(\kappa))^2{1\over \pi }
\sum_{\alpha\beta}(\delta_{\alpha\beta} 
- \hat Q_\alpha \hat Q_\beta)
Im \Phi^{\alpha,\beta}(\omega){-\beta \omega \over 1-e^{-\beta \hbar \omega}} 
$$
Here the scattering function depends only on the scattering vector
$\vec Q= \vec k - \vec k'$ and the energy loss $(\hbar)\omega =E(k)-E(k')$
Note that in our formulas $\omega$ contains a factor $\hbar$ and is the
energy loss. If we want to have meV as energy unit and Kelvin as temperature
unit, we have to write $\beta= 11.6/T$.
   
For the analysis of polarised neutron scattering the different
spin-components $S^{\alpha\beta}(\vec Q,\omega)$ of $S$ are needed. 
These are defined by
$$
S(\vec Q,\omega)= \sum_{\alpha\beta}(\delta_{\alpha\beta} 
- \hat Q_\alpha \hat Q_\beta)S^{\alpha\beta}(\vec Q,\omega)
$$
with
$$
S^{\alpha\beta}(\vec Q,\omega)=
Im \chi^{\alpha\beta}/(1-e^{-\beta \hbar \omega})=
Im \Phi^{\alpha,\beta}(\omega){-\beta \omega \over 1-e^{-\beta \hbar \omega}} 
$$
The complex dynamic susceptbility is calculated from
$$
\chi^{\alpha\beta}(\omega)= \chi^{\alpha\beta}(0)-\beta \omega
\Phi^{\alpha\beta}(\omega)= \sum_{\mu\nu}\beta (J^\alpha_\mu)^*(P_{\mu\nu}-\omega
\Phi_{\mu\nu}(\omega ))J^\beta_\nu
$$
where the static susceptibilities $\beta P_{\mu\nu}$ are diagonal in our
approximation.  

\vfill
\eject
\parindent=0pt
