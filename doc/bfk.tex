

%\documentclass[a4paper,12pt]{article}
%\usepackage{epsfig}

%\begin{document}
%\pagenumbering{arabic}
%\thispagestyle{empty}
%\setcounter{page}{1}

%{\Large
%\centerline
\section{{\prg bfk}\index{bfk} - Inelastic neutron-scattering from RE ions in a crystal field
%}
%\centerline{
including damping effects due to the exchange interaction with conduction
electrons}
%}

\bigskip
This is based on an extension  of the theory published by Klaus W. Becker, Peter Fulde and
Joachim Keller in Z. Physik B 28,9-18, 1977 
"Line width of crystal-field excitations in metallic rare-earth systems"
(for details see appendix~\ref{bfktheory}).
Here we give an  introduction to the computer program {\prg bfk} for the calculation of  the neutron 
scattering cross section. The computer program {\prg bfk} is written by J. Keller,
University of Regensburg.


{\large Description of the program:}

The program calculates the dynamical susceptibility
and the neutron scattering cross-section of single RE ions in the presence
of crystal fields and Landau damping due to the exchange interaction with 
conduction electrons.

It needs the following input-files (not all are needed for all tasks)

1. A file containing the information about the RE ion: Type of ion, number of
CF-levels, energy eigenvalues and eigenstates. The date are extracted from
the input-file by reading the information contained in lines starting with
$\#!$ or blanks, see the attached example. 

2. File with the formfactor data for RE ion

3. File with a list of ($\vec Q,\omega$)-values, for which the calculation
shall be performed

4. A parameter-file containing  the names of the files with the
formfactor, the  table with the ($\vec Q,\omega$)-values, the energy range, 
scattering direction etc., see the attached example.

5. The value of the coupling constant $g=j_{ex}N(0)$, the temperature,  
the mode of calculation, the form of the out-put, the name of the file with
the CEF-data, the  name of the parameter file are provided by the
commandline, which is used to start the program. 

   
The program consists of a number of modules and subroutines which are
briefly described in the following: 
 
1. Modules CommonData, MatrixElements, FormfactorPreparation

These modules contain definitions of global variables and arrays 
used in the program and in different  subroutines. 
FormfactorPreparation also contains the
subroutine FormfactorTransformation which transforms an input-file with
formfactor data into a file with formfactor values for equidistant Q-values. 
and the function Formfac to calculate the formfactor at
arbitrary Q-values.
  
2. Subroutine ReadData

Subroutine to read-in data needed to calculate the dynamical susceptibility
and the neutron scattering cross-section.

It reads the commandline, containing the coupling $g=j_{ex} N(0)$,  
the temperature  $T$ (in Kelvin), mode of calculation (see below), form of out-put, 
name of the
file with RE data, name of the parameter-file (containing also the name of
the file with the formfactor data). The information about the RE ion is
transferred into a workfile cefworkfile.dat for inspection and use in the
following runs. The data contained in the parameter-file are stored in the
file bfkdata.dat. The latter two have to be given only in the first run. If
they are left-out  in the following runs, the are assumed    to be
unchanged.

3. Subroutine Matrixelements

a) Calculates angular momentum matrices jjx, jjy, jjz for the crystal-field eigenstates
(2-dim arrays, dimension Ns x Ns). The three directional components are also
stored in the 3-dimensional array jjj(3,Ns,Ns). 

b) Calculates Boltzmann-factors $p(n)$. A cut-off in the exponent $\beta
E(n)$ is introduced such that Boltzmann factors with large negative
exponents
are set equal to zero. 

c) Defines  a set of transitions  $\nu$  between states
n1 and n2, stored in two  1-dim
arrays v1($\nu$), v2($\nu$). If both Boltzmann factors of the two states
involved are zero, this transition is eliminated from the set of allowed
transitions.  

d) Calculates static suscepibilities $P(\mu)$ for the standard basis operators
$K_{n,m}$ for the allowed transitions.  

e) All these reults are stored in a  file bfkmatrix.dat for examination, if
something goes wrong.
 

4. MatrixInversionSubroutine

adapted from Numerical Recipes, to be used for the inversion of the complex
matrix $\Omega_{\nu\mu}$. Called by 5. 

5. Subroutine Relmatrix

Calculates the matrix relaxation function $\Phi_{\mu,\nu}(\omega)$ for the set 
of dynamical variables
obtained from  the standard basis operators for a given energy (freqency)
$\omega$.

6. Subroutine Suscepcomponents 

Calculates the different components of the
dynamical susceptibility
$$
\chi^{\alpha,\beta}(\omega) = \beta\sum_{\mu \nu}(J^\alpha)_\mu)^*
\bigl[P_\mu\delta_{\mu\nu}-\omega \Phi_{\mu\nu}(\omega)\bigr]J^\beta_\nu
$$ 
and 
$$
Im \chi^{\alpha\beta}(\omega)/(1-\exp(-\beta\omega))
$$
7. Function Scatfunction(Q,$\omega$)

Calculates
$$
S(\vec Q,\omega)= \sum_{\alpha\beta}\bigl(\delta_{\alpha\beta} - \tilde
Q^\alpha\tilde Q^\beta\bigr) Im \chi^{\alpha\beta}/(1-\exp(-\beta\omega))
$$


8. Subroutine OutputResults 

Here the results for the dynamical susceptibility, the scattering function
and the differential neutron scattering cross section for different
scattering geometries are calculated, and the results written into files 
bfkm.res for different scattering-modes m=0-6, which are written into the  
subdirectory /results. Depending on the value of ms=1,2 the new results
over-write the previews results or append.  

Depending on the number m=0-6 (3. entry of the commandline) the following
results are calculated. 

mode=0: all nine components $\chi^{\alpha\beta}(\omega)$  of the complex 
dynamic susceptibility are calculated for $Npoint$ equidistant energies  $\omega$ 
between $emin$ and $emax$. 

mode=1: the diagonal components of $Im \chi^{\alpha\alpha}(\omega)/\tanh{\beta\omega/2}$
are calculated and the frequency integral is compared with the sum-rule 
$$
\sum_\alpha {1\over\pi}\int d\omega {Im
\chi^{\alpha\alpha}\over \tanh(\beta\omega/2)} 
= J(J+1) 
$$
The sum-rule sometimes is not very well fulfilled, since within this
approximation the Landau damping does not fall-off fast enough at large
energies

mode=2: The scattering function 
$$
S(\vec Q,\omega)=\sum_{\alpha,\beta} (\delta_{\alpha, \beta}- \tilde Q_\alpha \tilde
Q_\beta){ \chi''_{\alpha \beta}(\omega)\over 1-\exp(-\beta\omega)}
$$ 
is calculated for a given set of values for  
energy loss $\omega$ and scattering vector $\vec Q$ contained in a file
specified in the parameterfile. 

mode=3: The 9 different components of the scattering-function 
$$
(r_0g_JF(\vec Q)/2)^2{1\over \pi} S^{\alpha\beta}(\vec Q,\omega), \quad 
S^{\alpha\beta}(\vec Q,\omega)= {Im \chi^{\alpha,\beta}(\omega)\over
1-exp(-\beta\omega)}
$$
are  calculated. 

mode= 4-6: the neutron scattering cross section
$$
{d^2\sigma \over d\Omega d E'} =  {k' \over
k}S(\vec Q,\omega)
$$
with
$$
S(\vec Q, \omega)=({r_0\over 2}g F(Q))^2{1\over \pi }
\sum_{\alpha\beta}(\delta_{\alpha\beta} 
- \hat Q_\alpha \hat Q_\beta)
{Im \chi^{\alpha,\beta}(\omega)\over 1-exp(-\beta \omega)} 
$$

is  calculated for different scattering geometries:
In mode 4 the  
direction of the wave vector $\vec k$ and the energy $E=k^2/2m$ of the
incident 
beam is fixed. The direction of the scattering wave vector $\vec Q$ is fixed, 
but the 
length of $\vec Q$ is variable. The wave vector of the scattered
particles is  $\vec k'=\vec k-\vec Q$, their energy is $E'=k'^2/2m$ and
the energy loss is $\omega =E-E'$. In mode 5 the direction  of the wave vectors 
$\vec k$ and $\vec k'$ of the incoming and scattered beam
are fixed, while the energy $E'$ of the scattered beam is variable. In mode
6
the energy $E$ of the incident particles is variable 
and the energy $E'=k^2/2m$ of the scattered particles  fixed.  

\newpage
{\large How to run the program:}

\bigskip
The translated program is started with a command-line 
like 

bfk 0.1 10 0 1 prlevels.cef paramfile.par


with  the following
structure:

name of the program: bfk;  coupling constant g; temperature T (in K); type of
calculation: mode =1...6; type of output: mst=1 overwrite, mst=2 append new
results;  name of file with RE ion data; name of parameter file.

\medskip

The last two entries can be skipped in later runs, if they are not changed.

The mode number mode = 1 \dots 6
refers to the subject of calculation. The output- number 
mst=1,2 refers to the type of output-storage.

\medskip
The file with RE data should have the  form produced by sol1on (see the attached example). 
The
lines starting with numbers or blanks contain information, the lines
starting with \# are commentaries, the lines starting with  \#! also carry
information.

\medskip 
The parameterfile contains additional parameters needed to run the program:
energy range and number of energy values. Energies of incident or scattered
particles, direction of incident  or scattered particles. 

mode=4: E energy of incident particles, k11,k12,k13 direction if incident
particles (vector with arbitrary length), k21,k22,k23 direction of scattered
particles. 

mode=5: E energy of incident particles, k11,k12,k13 direction if incident
particles (vector with arbitrary length), k21,k22,k23 direction of
scattering vector $\vec Q$.

mode =6: E energy of scattered  particles, k11,k12,k13 direction if incident
particles (vector with arbitrary length), k21,k22,k23 direction of
scattered particles. 

\medskip
The parameterfile also contains the namme of a file with  a list of
scattering vectors  $\vec Q$ and energy loss ($\omega$) needed for mode
2,3

 Finally it contains the name of a file with the formfactor of the ion.  

\bigskip

J. Keller, May 2013

%\end{document}
