
%\documentclass[a4paper,12pt]{article}
%\usepackage{epsfig}

%\begin{document}
%\pagenumbering{arabic}
%\thispagestyle{empty}
%\setcounter{page}{1}

%{\Large
%\centerline
\section{Inelastic neutron-scattering from RE ions in a crystal field
%}}
%{\Large
%\centerline{
and calculation of the dynamical susceptibility}
%}
%\bigskip

\bigskip
This is an extension  of the theory published by Klaus W. Becker, Peter Fulde and
Joachim Keller in Z. Physik B
28,9-18, 1977 
"Line width of crystal-field excitations in metallic rare-earth systems"
and an  introduction to the computer program for the calculation of  the neutron 
scattering cross section. The computer program is written by J. Keller,
University of Regensburg.

\medskip

\noindent
Scattering cross section:
 
The neutron-scattering cross section is related to the dynamic susceptibility
of the spins of    the RE ions by 
(Stephen W. Lovesey; "Theory of neutron 
scattering from condensed matter"
Vol 2, equ. 11,144)



$$
{d^2\sigma \over d\Omega d E'}=  {k' \over
k}({r_0\over 2}g F(\kappa))^2{1\over \pi }
\sum_{\alpha\beta}(\delta_{\alpha\beta} 
- \hat \kappa_\alpha \hat \kappa_\beta)
{ \chi{"}_{\alpha,\beta}(\omega)\over 1-e^{-\beta \hbar \omega}} 
$$
where
$$
\chi_{\alpha,\beta}(z)=\int_{-\infty}^{+\infty} dt e^{izt}\chi_{\alpha\beta}(t), \quad z=\omega
+i\delta 
$$
is the Fourier-Laplace transform of the spin susceptibility
$$
\chi_{\alpha\beta}(t)={i\over \hbar} \Theta(t)\langle [J^\dagger_\alpha(t), J_\beta(0)]\rangle 
$$
Here $k$ and
$ k'$ denote the  wave number of the neutron before and after the
scattering. $\vec \kappa = \vec k' - \vec k$ is the scattering wave vector,
$\tilde \kappa = \vec \kappa/\vert\vec\kappa\vert$. 
$r_0= -0.54 \cdot 10^{-12}$ cm, $g_l$ is the Land\'e factor, 
$F(\kappa)$ the atomic form factor of the
rare earth ion.
  
\bigskip
\noindent
The model:

We calculate the spin susceptibility of a RE ion in the presence of exchange
interaction with conduction electrons. The system is described by the
Hamiltonian
$$
H=H_{cf}+H_{el}+H_{el,cf} 
$$
The first part is the cf-Hamiltonian of the spin-system: 
$$ 
H_{cf}= \sum_n E_n K_{nn}, \quad K_{nm}= \vert n\rangle \langle m\vert
$$  
written in terms of the crystal field eigenstates $\vert n\rangle$. 
The second part is the Hamiltonian of the conduction electrons
$$
H_{el}=\sum_{k\alpha}\epsilon_kc^\dagger_{k\alpha}c_{k\alpha}
$$
The third part is the 
 interaction between local moments and  the conduction electrons
$$
H_{el,cf}= - J_{ex}\vec J \cdot \vec \sigma, \quad \vec \sigma = \sum_{k\alpha
q\beta}\vec \sigma_{\alpha\beta}c^\dagger_{k\alpha}c_{k+q\beta}, \quad \vec
J=\sum_{n,m}\vec J_{n,m}K_{nm}.
$$
We assume, that the energies $E_n$ and the eigenstates $\vert n\rangle$
expressed by angular momentum eigenstates are known.

For the calculation of the dynamic susceptibility we use the Mori-Zwanzig
projection formalism:
For a set of dynamical variables $A_i$ we define the susceptibilities 
$$
\chi_{i,k}(t)=i \Theta(t) \langle [A_i^\dagger(t),A_k(0)]\rangle
$$ 
and their Laplace transform
$$
\chi_{i,k}(z)=i\int_0^\infty dt e^{izt} \langle [A_i^\dagger(t),A_k(0)]\rangle
$$ 
(From here on we skip factors of $\hbar$ which drop out from the final
result if we replace the frequencies $\omega$ by the correponding energies
$\hbar \omega$.  The corresponding static (isothermal) susceptibilities are given by
$$
\chi_{i,k}(0) = \int_0^\beta d \lambda \langle e^{\lambda H}
A_i^\dagger e^{-\lambda H}  A_k\rangle
$$
With help of the Liouvillian ${\cal L}$ acting on operators as 
$$
{\cal L}A=[H,A]
$$ 
the time dependence of an operator can be expressed by
$$
A(t)=e^{iHt}Ae^{-iHt}= e^{i{\cal L}t}A, \quad e^{\lambda H}Ae^{-\lambda H}=
e^{\lambda {\cal L} }A
$$
and the static and dynamical susceptibilities can be written as
$$
\chi_{i,k}(0) = \int_0^\beta d \lambda \langle (e^{\lambda {\cal L}}
A_i^\dagger)   A_k\rangle
$$
$$
\chi_{i,k}(z)=i\int_0^\infty dt e^{izt} \langle [e^{it{\cal
L}}A_i^\dagger,A_k(0)]\rangle= i\int_0^\infty dt e^{izt} \langle [A^\dagger,e^{-it{\cal
L}}A_i(0)]\rangle
$$ 
which can be formally integrated to yield
$$
\chi_{i,k}(z)= -\langle [A_i^\dagger,{1\over {z-\cal
L}}A_k(0)]\rangle
$$ 
The static susceptibilities are used to define a scalar product between  the
dynamical variables:
$$
(A_i \vert A_k) =  {1\over \beta }\int_0^\beta d\lambda \langle
(e^{\lambda{\cal L}}A_i^\dagger)A_k\rangle ={1\over \beta} \chi_{ik}(0)
$$
It fulfills the axioms of a scalar product and  furthermore it has the important property
$$
({\cal L}A_i\vert A_k)=(A_i\vert {\cal L}A_k)={1\over \beta}\langle
[A_i^\dagger,A_k]\rangle
$$
With help of this relation the dynamical susceptibility can be
expressed as
$$
\chi_{i,k}(z)= -\beta (A_i\vert {{\cal L}\over {z-\cal L}} A_k)
$$
and finally as 
$$
\chi_{ik}(z)=\chi_{ik}(0)-z\beta (A_i\vert {1\over {z-\cal L}}A_k)  
$$
The second term is the so-called relaxation function 
$$
\Phi_{ik}(z)=(A_i \vert {1 \over {z-\cal L}}A_k)
$$

The idea of the projection formalism to calculate the dynamical
susceptibility of a variable $A$ is to project this variable  onto a closed
set of dynamical variables $A_i$ and to solve approximately the coupled
equations between these variables. For this purpose a projector is defined
by
$$
{\cal P} A= \sum_{\nu \mu}A_\nu P^{-1}_{\nu \mu}(A_\mu\vert A) \quad
P_{\nu\mu}=(A_\nu\vert A_\mu)
$$ 
where $ P^{-1}_{\nu \mu}=[P^{-1}]_{\nu \mu}$ is the ${\nu\mu}$-component of the inverse matrix of
$P$. 

For the  resolvent operator of the relaxation function
$$
{\cal F}(z)= {1\over {z-\cal L}}, \quad ({z-\cal L}){\cal F}(z)=1 
$$
one obtains the exact equation
$$
({\cal P}(z-{\cal P}{\cal L}{\cal P} - {\cal P} {\cal M}(z) {\cal P}){\cal P} {\cal
F}(z) {\cal P}= {\cal P}
$$
with the memory function
$$
{\cal M}(z)={\cal PLQ}{1\over z-{\cal QLQ} }{\cal QLP}
$$ 
where ${\cal Q}=1-{\cal P}$. In components
$$
\Phi_{\nu\mu}(z)= (A_\nu\vert {1\over z-{\cal L}} A_\mu)
$$
$$
\sum_\lambda \Bigl(z\delta_{\nu\lambda}-\sum_\kappa\bigl[L_{\nu\kappa}+
M_{\nu\kappa}(z)\bigr]P^{-1}_{\kappa\lambda}\Bigr)\Phi_{\lambda\mu}(z)
=P_{\nu \mu}
$$
with
$$
L_{\nu\mu}=(A_\nu\vert {\cal L}A_\mu)
$$
and the memory function 
$$
M_{\nu\mu}(z)=(A_\nu\vert{\cal M}(z)A_\mu)
$$

Now we apply the formalism to the coupled spin-electron system and restrict
ourselves to the lowest order contributions of the spin electron
interaction. As dynamical variables we choose a decomposition of the  
original spin-variable:
$$
J^\alpha=\sum_{n_1,n_2}J^\alpha_{n_2,n_1}K_{n_2,n_1}=\sum_\nu J^\alpha_\nu A_\nu, 
\quad 
A_\nu= K_{n_1n_2}   
$$
where $\nu$ denotes a transition $n_2 \gets n_1$ performed with the
standard-basis operator $\vert n_2\rangle\langle n_1\vert$. 

In lowest (zeroth) order in the el-cf interaction  
$$
{\cal L}A_\nu = (E_{n_2}-E_{n_1)}A_\nu
$$
and  the scalar product is diagonal in lowest order in the transition
operators,
$$
P_{\nu\mu}=(A_\nu\vert A_\mu)\simeq\delta_{\nu
\mu}P_\nu, \quad P_\nu=(A_\nu\vert
A_\nu)={p(n_1)-p(n_2)\over \beta (E_{n2}-E_{n_1})}
$$
where $p(n)=\exp(-\beta E_n)/Z$ is the thermal occupation number. For the
frequency term we then get 
$$
L_{\nu\mu}=\delta_{\nu\mu}(A_\nu\vert A_\nu) (E_{n_2}-E_{n_1} )
+O(J_{ex}^2)
$$
Neglecting the second-order energy corrections in the following we obtain
the equation for the relaxation function 
$$
\Phi_{\nu\mu}(z)=\bigl[\Omega^{-1}\bigr]_{\nu\mu}(z)
P_\mu, \quad
\Omega_{\nu\mu}(z)=(z-E_\nu)\delta_{\nu\mu} -
M_{\nu\mu}(z)[P^{-1}]_\mu, \quad
E_\nu = E_{n_2}- E_{n_1}
$$
and it remains to calculate the memoryfunction containing the relaxation
processes. 

In lowest order in the electron-spin interaction ${\cal QL}A_\nu$ can be replaced by ${\cal L}_{el,cf}A_\nu$.
Then we get for the memory function
$$
M_{\nu \mu}(z)= ({\cal L}_{el,cf}A_\nu \vert{1\over z- {\cal
L}_0}{\cal L}_{el,cf}A_\mu)=
M_{n_2n_1,m_2m_1}(z)
$$
with
$$
M_{n_2n_1,m_2m_1}(z)=({\cal L}_{el,cf} K_{n_2n_1}\vert{1\over z-{\cal
L}_0}{\cal L}_{el,cf} K_{m_2m_1})
$$

Now
$$
{\cal L}_{el,cf} K_{n_2n_1} = J_{ex}\sum_t \vec \sigma(\vec  J_{n_1t}
K_{n_2t} - \vec
J_{tn_2}K_{tn_1})
$$
with
$$
\vec \sigma = \sum_{k\alpha, k+q\beta}\vec \sigma_{\alpha\beta}
c^\dagger_{k\alpha}c_{k+q\beta}
$$
With help of the symmetry properties 
$$
( \sigma^i K_{nm}\vert {1\over z- {\cal L}_0}\sigma^j K_{n'm'})=
\delta_{ij}\delta_{nn'}\delta_{mm'}G_{nm}(z)
$$
with
$$
G_{nm}(z)=( \sigma^i K_{nm}\vert {1\over  {z-\cal L}_0}\sigma^i K_{nm})
$$
we obtain
\eqnarray
M_{n_2n_1,m_2m_1}(z)=J_{ex}^2\sum_i\Bigl[&
\delta_{n_2m_2}\sum_tJ^i_{m_1t}J^i_{tn_1}G_{n_2t}
+ \delta_{n_1m_1}\sum_tJ^i_{
n_2t}J^i_{tm_2}G_{tn_1}\nonumber \\
&-J^i_{m_1n_1}J^i_{n_2m_2}G_{n_2m_1}
-J^i_{n_2m_2}J^i_{m_1n_1}G_{m_2n_1}\Bigr]\nonumber
\endeqnarray


In order to calculate the relaxation functions $G_{n,m}(z)$
we use the general relation between relaxation function and dynamic
susceptibility
$$
\chi(z)= \chi(0)-\beta z \Phi(z)
$$
and calculate instead the corresponding susceptibility (using tr
$\sigma^i\sigma^i)=2$):
\eqnarray
G_{nm}(z)&= {2\over \beta \omega}\sum_{k,k+q} \langle \Bigl[K_{mn}
 c^\dagger_{k+q}c_{k},(z - E_n +
E_m  -\epsilon_k+\epsilon_{k+q})^{-1} K_{nm}
c^\dagger_{k}c_{k+q}\Bigr]\rangle \nonumber\\
&=  {2\over \beta \omega
}\sum_{k,q}(f_{k+q}(1-f_{k})p_m-f_{k}(1-f_{k+q})p_n)(
z-E_n+E_m-\epsilon_{k}+\epsilon_{k+q})^{-1}\nonumber
\endeqnarray
We are interested in the imaginary part describing the relaxation processes:
$$
Im G_{nm}(\omega+i\delta)= - {2\pi \over \beta \omega
}\sum_{k,q}\Bigl(f_{k+q}(1-f_{k})p_m-f_{k}(1-f_{k+q})p_n \Bigr)
 \delta(\omega -E_n+E_m-\epsilon_{k}+\epsilon_{k+q})
$$
Writing $ \rho=\omega - \omega_{nm}$ and $\omega_{nm}=E_n-E_m$  we obtain
$$
Im G_{nm}(\omega+i\delta)= - {2\pi N^2(0)\over \beta \omega} \int d\epsilon  
 (f(\epsilon)(1-f(\epsilon+\rho) p_m
- f(\epsilon+\rho)(1-f(\epsilon))p_n)
$$
For the integrals we get
\eqnarray
\int d\epsilon f(\epsilon)(1-f(\epsilon+\rho)=&\int d\epsilon
\exp(\beta(\epsilon+\rho))/
(1+\exp(\beta\epsilon))(1+\exp(\beta(\epsilon+\rho))\nonumber\\
=& 
(\omega-\omega_{nm})\exp(\beta(\omega-\omega_{nm}))/
(-1+\exp(\beta(\omega-\omega_{nm})\nonumber
\endeqnarray
\eqnarray
\int d\epsilon f(\epsilon+\rho )(1-f(\epsilon)=&\int d\epsilon
\exp(\beta(\epsilon)/
(1+\exp(\beta\epsilon))(1+\exp(\beta(\epsilon+\rho))\nonumber\\
=& 
(\omega-\omega_{nm})/
(-1+\exp(\beta(\omega-\omega_{nm}))\nonumber
\endeqnarray
This makes
$$
Im G_{nm}=-{2\pi N^2(0)\over \beta \omega}(\omega -\omega_{nm}) {1-\exp(-\beta
\omega)\over
1-\exp[(\omega_{nm}-\omega)\beta]}p_m
$$
which has to be used to calculate the imaginary part of the memory function.
Writing 
$$
F_{nm}(\omega )= {1\over \beta \omega}(\omega -\omega_{nm}) {1-\exp(-\beta
\omega)\over
1-\exp[(\omega_{nm}-\omega)\beta]}p_m
$$
which also be written in symmetrized form as
$$
F_{nm}(\omega )= {\sqrt{p_np_m}\over \beta}{(\omega -\omega_{nm})\over
\omega} {\exp(\beta \omega)/2) - \exp(-\beta \omega)/2)\over
\exp(\beta (\omega-\omega_{nm})/2) - \exp(-\beta (\omega -\omega_{nm})/2)}
$$
we obtain with $g=J_{ex}N(0)$
\eqnarray
M_{n_2n_1,m_2m_1}(\omega) =- i 2\pi g^2
\sum_i\Bigl[&
\delta_{n_2m_2}\sum_tJ^i_{m_1t}J^i_{tn_1}F_{n_2t}
+ \delta_{n_1m_1}\sum_tJ^i_{
n_2t}J^i_{tm_2}F_{tn_1}\nonumber\\
&-J^i_{m_1n_1}J^i_{n_2m_2}F_{n_2m_1}
-J^i_{n_2m_2}J^i_{m_1n_1}F_{m_2n_1}\Bigr]
\nonumber
\endeqnarray
from which we get the memory function matrix in the space of dynamical variables
$$
M_{\nu \mu}(\omega)= M_{n_2n_1,m_2m_1}(\omega)
$$

\bigskip
\noindent


Summary:
For the neutron scattering cross section we need the function 
$Im u^{\alpha\beta}(\omega+i\delta)/(1-\exp(-\beta\omega)$, where
$u^{\alpha\beta}(z)$ is the frequency dependent part of the dynamic
susceptibility $\chi^\alpha\beta(z)$ for spin components  $J^\alpha$,$J^\beta$, which is
related to the corresponding relaxation function $\Phi^{\alpha,\beta}$ by
$$
u^{\alpha\beta}(z)= -\beta z \Phi^{\alpha\beta}(z)
$$
For the full dynamical susceptibility we need also the static suseptibility
$$
\chi^{\alpha\beta}(z) = \chi^{\alpha\beta}(0) - \beta z \Phi^{\alpha\beta}(z)  
$$
where in lowest order in the exchange interaction 
$$
\chi^{\alpha\beta}(0)  = \sum_\nu \beta (J^\alpha_\nu)^\dagger P_\nu J^\beta_\nu
$$
The above relaxation function is calculated with help of the Mori-Zwanzig
projection formalism by 
$$
\Phi^{\alpha\beta}(z)=\sum_{\mu\nu}
(J^\alpha_\nu)^*\Phi_{\nu\mu}(z)J^\beta_\mu
$$
where $\nu$ denotes a transition from $n_1$ to $n_2$ between crystal field
levels of the magnetic ion. The partial relaxation functions are obtained by 
solving the matrix
equation
$$
\Phi_{\nu\mu}(z)= [\Omega^{-1}]_{\nu\mu}P_\mu      
$$
with
$$
\Omega_{\nu\mu}(z)= (z-\omega_\nu)\delta_{\nu\mu}  -M_{\nu\mu}(z)/P_\mu
$$
where $\omega_\nu =E_{n_2}- E_{n_1}$ is the energy difference of cf-levels.

Only terms in  lowest  order in the el-ion interaction are kept. We neglect
frequency shifts due to the electron-ion interaction. 
Then the  memory function is purely
imaginary (with a negative  sign).

Note that compared to our paper BFK, Z.Physik B28, 9-18, 1977 we have used here 
a different sign-convention.

For numerical reasons it is more convenient to calculate the relaxation
function in the following way:
$$
\Phi_{\nu\mu}(z)= P_\nu[\bar\Omega^{-1}]_{\nu\mu}P_\mu      
$$
with
$$
\bar \Omega_{\nu\mu}(z)= P_\nu(z-\omega_\nu)\delta_{\nu\mu}  - M_{\nu\mu}(z)
$$

From the relaxation function we get for the dynamic scattering cross section 
$$
{d^2\sigma \over d\Omega d E'} =  {k' \over
k}S(\vec \kappa,\omega)
$$
with
$$
S(\vec \kappa, \omega)=({r_0\over 2}g F(\kappa))^2{1\over \pi }
\sum_{\alpha\beta}(\delta_{\alpha\beta} 
- \hat \kappa_\alpha \hat \kappa_\beta)
Im \Phi^{\alpha,\beta}(\omega){-\beta \omega \over 1-e^{-\beta \hbar \omega}} 
$$
Here the scattering function depends only on the scattering vector
$\vec \kappa= \vec k - \vec k'$ and the energy loss $(\hbar)\omega =E(k)-E(k')$
Note that in our formulas $\omega$ contains a factor $\hbar$ and is the
energy loss. If we want to have meV as energy unit and Kelvin as temperature
unit, we have to write $\beta= 11.6/T$.
   
For the analysis of polarised neutron scattering the different
spin-components $S^{\alpha\beta}(\vec \kappa,\omega)$ of $S$ are needed. 
These are defined by
$$
S(\vec \kappa,\omega)= \sum_{\alpha\beta}(\delta_{\alpha\beta} 
- \hat \kappa_\alpha \hat \kappa_\beta)S^{\alpha\beta}(\vec \kappa,\omega)
$$


\bigskip
\newpage

\vfill
\eject
\parindent=0pt
{\large Construction of the program for single-ion dynamical susceptibility:}


1. Module CommonData, module MatrixElements

Contains definitions of global variables and arrays used in the program and
in different  subroutines 

2. Subroutine ReadData

Subroutine to read-in data about the magnetic ions, the coupling 
constant $g=J_{ex} N(0)$ and the temperature  T (in Kelvin). 
It is assumed that the data about the
magnetic ions (number of of states Ns, Lande factor, energies of the 
CEF states and 
eigenvectors) are contained in an input-file provided by other 
programs of McPhase, in particular from so1ion. The program picks-up the
required data by searching lines beginning with \#! or blanks. Furthermore a file
containing information about the energy range of the calculation is
required. All these informations are entered via the commandline of the
executable of the program. For a detailed prescription to run the program see below.


3. Subroutine Matrixelements

a) Calculates angular momentum matrices jjx, jjy, jjz for the crystal-field eigenstates
(2-dim arrays, dimension Ns x Ns). The three directional components are also
stored in the 3-dimensional array jjj(3,Ns,Ns). 

b) Calculates Boltzmann-factors $p(n)$. A cut-off in the exponent $\beta
E(n)$ is introduced such that Boltzmann factors with large negative
exponents
are set equal to zero. 

c) Defines  a set of transitions  $\nu$  between states
n1 and n2, stored in two  1-dim
arrays v1($\nu$), v2($\nu$). If both Boltzmann factors of the two states
involved are zero, this transition is eliminated from the set of allowed
transitions.  

d) Calculates static suscepibilities $Pp(\mu)$ for the standard basis operators
$K_{n,m}$ for the allowed transitions.  

e) All these reults are stored in a  file bfkmatrix for examination, if
something goes wrong.
 

4. MatrixInversionSubroutine

adapted from Numerical Recipes, to be used for the inversion of the complex
matrix $\Omega_{\nu\mu}$. Called by 5. 

5. Subroutine SuscepComponents

Calculates the matrix relaxation function for the set of dynamical variables
obtained for the standard basis operators for a given energy (freqency)
$\omega$ and multiplies them with the corresponding spin-matrix-elements to
get the complex dynamical susceptibility of the RE ion.  

6. Subroutine OutputResults 

Here the results for the dynamical susceptibility is calculated for
different energies and printed into a file chibfk.res in the subdirectory
/Results.



\newpage
{\large How  to run the program chibfk\index{chibfk}}.

The program is started by typing-in the name of the  executable program (chibfk). 
%If the
%program is translated by gfortran, this name would be a.out, but this name can
%be changed to another name. 
Via the commandline the value of the dimensionless coupling 
constant g with the conduction electrons and the temperature T (in Kelvin) is
entered. The next entry is the name of the file containing information about
the level structure of  the RE ions (for instance the output levels.cef 
from so1ion). Finally the name
of a file (for instance mcdisp.par), containing information about 
the energy range emin, emax 
of  the calculation is required. These files should be in the same directory 
as the executable program.
Then the commandline has the form: 

chibfk 0.02 20 levels.cef mcdisp.par

Results are calculated for the different directionl components of the
dynamical susceptibility and written into a file /Results/chibfk.res in a
subdirectory Results. 
 
Joachim Keller, April 2013 


\vfill
\eject
\parindent=0pt
{\large Construction of the program for the neutron scattering:}


1. Module CommonData, module MatrixElements, module Ffp

Contains definitions of global variables and arrays used in the proram and
in different  subroutines 

2. Subroutine ReadData

Subroutine to read-in data about the magnetic ions, the coupling 
constant $g=J_{ex} N(0)$ and the temperature  T (in Kelvin). 
It is assumed that the data about the
magnetic ions (number of of states Ns, Lande factor, energies of the 
CEF states and 
eigenvectors) are contained in an input-file provided by other 
programs of McPhase, in particular from so1ion. Also required is a file
containing the formfactor $F(Q)$. For one mode of calculation a list of
Omega and Q values is required, for wihch the cross section shall be
calculated.  For a detailed prescription to run the program see below.


3. Subroutine Matrixelements

a) Calculates angular momentum matrices jjx, jjy, jjz for the crystal-field eigenstates
(2-dim arrays, dimension Ns x Ns). The three directional components are also
stored in the 3-dimensional array jjj(3,Ns,Ns). 

b) Calculates Boltzmann-factors $p(n)$. A cut-off in the exponent $\beta
E(n)$ is introduced such that Boltzmann factors with large negative
exponents
are set equal to zero. 

c) Defines  a set of transitions  $\nu$  between states
n1 and n2, stored in two  1-dim
arrays v1($\nu$), v2($\nu$). If both Boltzmann factors of the two states
involved are zero, this transition is eliminated from the set of allowed
transitions.  

d) Calculates static suscepibilities $Pp(\mu)$ for the standard basis operators
$K_{n,m}$ for the allowed transitions.  

e) All these reults are stored in a  file bfkmatrix for examination, if
something goes wrong.
 

4. MatrixInversionSubroutine

adapted from Numerical Recipes, to be used for the inversion of the complex
matrix $\Omega_{\nu\mu}$. Called by 5. 

5. Subroutine SuscepComponente

Calculates the matrix relaxation function for the set of dynamical variables
obtained for the standard basis operators for a given energy (freqency)
$\omega$ and multiplies them with the coresponding spin-matrix-elements.

6. Subroutine Strfak(Q)

Calculates atomic formfactor from a list of formfactor values by
interpolation. 

7. Real Function Scatfunction( kappa,omega) 

Calculates the spin-dependend components of the dynamical scattering
cross section for given energy loss x, and multiplies them with the
directional part of the cross-section, uses SuscepComponents.
Calculates 
$\sum_{\alpha,\beta} (\delta_{\alpha, \beta}- \tilde \kappa_\alpha \tilde
\kappa_\beta) \chi''_{\alpha \beta}(\omega)/(1-\exp(-\beta\omega))$
for a given energy loss $\omega$ and scattering vector $\kappa$.


9. Subroutine OutputResults 

Here the results for the scattering cross section are printed into the
output file. 
6 different calculational modes $0,1,\dots ,5$ can be used. In mode 0 the
freqency dependent part of the dynamical cross section is analysed and $Im
\chi^{\alpha\alpha}(\omega)/\tanh{\beta\omega/2}$ is calculated. This
quantity should fulfill a sumrule 
$$
\sum_\alpha {1\over\pi}\int d\omega Im \chi^{\alpha\alpha}/\tanh(\beta\omega) 
= J(J+1) 
$$
In mode 1 and 2 the components of the cross section are calculated,
using the form factor $F(Q)$ as input. In mode 1 the
scattering function $S(\vec Q,\omega)$ is calculated for a set of $\vec
Q$ and $\omega $ values specified by an input file. In mode 2 the same
is done for the 9 different spin components $S^{\alpha\beta}$ separately 

In modes 3-5
the complete scattering cross section is calculated for different scattering
geometries: In mode 3 the  
direction of the wave vector $\vec k$ and the energy $E=k^2/2m$ of the incoming 
beam is fixed. The direction of the scattering wave vector $\vec q$ is fixed, 
but the 
length of $\vec q$ is variable. The wave vector of the scattered
particles is  $\vec ks=\vec k-\vec q$, their energy is $Es=ks^2/2m$ and
the energy loss is $\omega =E-Es$. In mode 4 the direction  of the wave vectors 
$\vec k$ and $\vec ks$ of the incoming and scattered beam
are fixed, while the energy $Es$ of the scattered beam is variable. In mode 5
the energy $E$ of the incoming beam is variable 
and the energy $Es=k^2/2m$ of the scattered particles  fixed.  




\bigskip

{\large How  to run the program bfk\index{bfk}}.

It is generally assummed that the necessary data 
about the magnetic ion are contained  in an input-file provided by other
parts of McPhase, in particular the file levels.cef from so1ion. 
This input-file must provide the 
following information: Type of rare-earth ion  (the J-value and Lande-factor
are derived from this information), 
energies and eigenstates of the ion in the crystalline electric field. 
This information is stored in a workfile cefworkfile.dat for later use. 

Also
information about the atomic formfactor must be provided, for instance by the
file 'formfactor.out' generated by the program formfactor. The essential
information contained on this file is transfered to the workfile    
strworkfile.dat. 

These two input files are read-in by the program, if it is started with an
empty command-line. Then it asks for the name of the two files.   
These names of the files containing the
CEF-data and the formfactor are stored in the files bfkdata0.dat (for mode
0), bfkdata12.dat (for modes
1 and 2) and
bfkdata345.dat (for modes 3-5).  
 bfkdata12.dat also contains the name of the file containing
the list of $q$-values and energy-loss values for which the scattering
function will be calculated.    

Further input is needed  for the scattering geometry for the modes 3-5 which
will be stored in bfkdata345.dat. This is the direction of the wave vectors 
$\vec k$, $\vec k'$ and(or)
$\vec q$, which are read-in as vectors of arbitrary length. Furthermore
the energy  of the incoming beam (mode 3 and mode 4) or of the scattered beam
(mode 5) is needed. 

Finally the value 
of the temperature $T$ (in Kelvin) and the value of the (dimensionless) 
coupling constant $g$ with the
conduction electrons are needed. They are asked for, if the program is
started with the name of the executable program and an empty command line 
 together with the number (mode=$0,1,\dots ,5$) of the
calculation mode and the type of storage of the results. Here one has the
choice between overwriting the results of the previous run in the output
file (mst=1), or to append the
new results to the previous results in the output file (mst=2). 
The names of the files containing the results of the calculation are
bfkresults0.dat bfkresults12.dat and bfkresults345.dat.

In the following runs  the values of g and T, mode of calculation, and type
of output can be  provided by the
command-line by starting the program with a filled-in command line: 
 a.out (name of executable ) 
g T mode mst. In these runs the workfiles cefworkfile.dat and
strworkfile.dat are used instead of the original files (containing
also commentary text lines). 

The program can also be run without the original
files, if the workfiles have been filled-in with the correct information and
have the correct
format. The program is then started with a filled-in commandline. Missing
data are asked for by the program.     


Joachim Keller, March 2013 

%\end{document}
\newpage

