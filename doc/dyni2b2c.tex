\newpage
\section{DyNi$_2$B$_2$C - single ion module}\label{dyni2b2c}

The crystal field ground state of the $J=15/2$ Dy$^{3+}$ ion in DyNi$_2$B$_2$C 
can be   described by a quasi-quartet consisting of two doublets separated
by a energy interval $\Delta$.

In order to calculate efficiently the single ion property (for small effective
magnetic fields in comparison to the total CF splitting) the 
Hamiltonian $H=H_{cf}- g_J \mu_b {\mbf H}{\mbf J}$ is projected to the quasi quartet
and may be written as

\begin{equation}\label{ham}
H=\left (
\begin{array}{cccc}
-\Delta/2 & 0 & 0 & 0 \\
0 & -\Delta/2 & 0 & 0 \\
0 & 0 & \Delta/2  & 0 \\
0 & 0 & 0 & \Delta/2 
\end{array}
\right)
 -g_J \mu_B (H_a J_a + H_b J_b + H_c J_b)
\end{equation}

with the angular momentum operators given by the  4x4 matrices
\begin{equation}
J_a=
\left (
\begin{array}{cccc}
0 & b & 0 & c \\
b & 0 & c & 0 \\
0 & c & 0 & e \\
c & 0 & e & 0 
\end{array}
\right )
\end{equation}

\begin{equation}
J_b=
\left (
\begin{array}{cccc}
0 & -ib & 0 & ic \\
+ib & 0 & -ic & 0 \\
0 & +ic & 0 & -ie \\
-ic & 0 & +ie & 0 
\end{array}
\right )
\end{equation}


\begin{equation}
J_c=
\left (
\begin{array}{cccc}
+a& 0 & 0 & 0 \\
0 &-a & 0 & 0 \\
0 & 0 &-d & 0 \\
0 & 0 & 0 &+d 
\end{array}
\right )
\end{equation}

The constants $a$-$e$ can be computed from the crystal field parameters, if
these are known. On the other hand, they are connected to the saturation
magnetic moments $\mbf M$ by the following equations ($\max(...)$ denotes the
maximum of the argument values)

\begin{equation}
M_{001}=g_J \mu_B \max(|a|,|d|) 
\end{equation}

\begin{eqnarray}
M_{100}&=&g_J \mu_B \lambda\\
2\lambda^2&=&e^2+2c^2+b^2 + \sqrt{(e^2+2c^2+b^2)^2-4(be{\mbf-}c^2)}
\end{eqnarray}

\begin{eqnarray}
M_{110}&=&g_J \mu_B  \lambda\\
2\lambda^2&=&e^2+2c^2+b^2 + \sqrt{(e^2+2c^2+b^2)^2-4(be{\mbf+}c^2)}
\end{eqnarray}

The module given in {\prg /examples/dyni2b2c/1ion\_mod/quartett.c} diagonalises
the Hamiltonian (\ref{ham}) and calculates the thermal expectation 
value $<>_T$ of the vector $\mbf J$,which
is returned to the {\prg McPhas} program.
