\section{{\prg ic1ion\index{ic1ion}} - a module for intermediate coupling}\label{ic1ion}


This powerful module, written by Duc Le, is able to do crystal field calculations
using intermediate coupling schemes. 

\subsection{using {\prg ic1ion\index{ic1ion}} as a stand-alone program to do crystal field calculations}

{\prg ic1ion} can be started by issuing the command {\prg ic1ion} (reading input from single ion %%@
parameter file {\prg mcphas.ic}). Alternatively, the single ion parameter file can be specified on the commandline as

{\prg ic1ion filename}

{\prg ic1ion} also accepts these options: 
\begin{description}
\item[{\prg -h}] which outputs in the file {\prg results/ic1ion.out} only
the header containing the free ion (Coulomb and spin-orbit) and single ion (crystal field and magnetic fields)
parameters. 
%\item[{\prg -t energylimit}] which calculates the output energy levels only up to the value of energy limit
%by truncating the free ion Hamiltonian at this limit, and calculating the crystal field and Zeeman terms only
%for free ion states below this limit.
\end{description}
The single ion parameter file has to be of the form (note the crystal field parameters in Wybourne notation
can be calculated by {\prg pointc}\index{pointc}):
{\footnotesize
\begin{verbatim}
#!MODULE=ic1ion
#<!--mcphase.sipf-->
# single ion property file example for module ic1ion

#-------------------------------------------------------------------------------------------
# OBLIGATORY PARAMETERS
#------------------------------------------------------------------------------------------

# Configuration *********************************************
# (i) either specify single ion type by IONTYPE 

# IONTYPE=Co2+

# (ii) or by giving parameters explicitly:
# from Griffith 1971 p 487  B=1115  C=4366 ZETA=533
# p379 CoIII=Co2+ is 3d7 configuration
conf = d7
F0=0
F2=1738.7
F4=124.74
zeta=533
# from equations p 83 A=F0-49*F4  B=F2-5*F4 C=35*F4
# we get F4=C/35  F2=B+5*F4
# however F0 is not determined from this table ...
#*******************************************************

# for charges and chrgplt to work you need to specify the number of electrons in the shell

nof_electrons=7

#-------------------------------------------------------------------------------------------
# OPTIONAL PARAMETERS
#-------------------------------------------------------------------------------------------

# some keywords, each in a separate line (each keyword sets a flag to true and holds
# for the rest of the file to be read)

# --- add a magnetic field (Tesla) ---
Bx=1
By=2
Bz=0.1

# -- choose the basis to be used for output of eigenstates in ic1ion.out ----
# Either:
#
# basis=mSmL
#
# For |L,S,mS,mL> basis. Or
 
 basis=JmJ

# For |L,S,J,mJ> basis (default if none specified).

# to allow output of more than 4 components of the eigenvector in ic1ion.out, 
# you need to specify the number X of eigenvectors here:

# eigenvectors=X

# --- some faster algorithms for matrix diagonalisation ...

# uncomment the next line to save matrices which are often used the first time they are
# calcultated in the directory results/mms, and then reread for subsequent instances.

# save_matrices

# if the next line is uncommented then it will use the Arnoldi method to calculate only 
# a few of the lowest eigenvalues in Icalc() [those for which the thermal occupation
# factor at 300K is nonzero]. Note: this method may sometimes give erroneous eigenvectors
# due to numerical instabilities. In addition, it may be disabled in Mac OS X versions.

# arnoldi

# Another more reliable method to find only a few of the lowest eigenvalue uses the 
# relatively robust representation algorithm in Lapack and can be accessibly by 
# uncommenting the keyword: "partial" below. This method takes about 2/3 the
# time of the full computation of Icalc(), whereas the arnoldi routine takes about
# 1/5 the time.

# partial

# Finally the full Hamiltonian matrix may be truncated at some levels, leaving only
# the lowest lying energy levels to be used in the calculation of the moments.
# This option is activated by uncommenting the line below, specifying a ratio X of
# levels to keep. For example, if X=0.2, for f2, 20% of the full 91 levels == 18
# levels will be used for the Icalc() calculation.
# This routine first calculates and diagonalises the singleion Hamiltonian without 
# the Zeeman or mean-field terms, and then uses these eigenvectors to make a unitary 
# transformation matrix to rotate the multipolar and L and S operators into this
# new basis. Then it truncates the matrix, leaving only the X fraction lowest 
# levels.
# This method is much faster (for low values of X) even than the arnoldi method, but
# is much more approximate. Note The most computationally extensive operations are 
# carried out at the start - the diagonalisation of the single ion Hamiltonian, and
# matrix-matrix multiplication to rotate the operator matrices. Once done the rotated 
# matrices are stored in memory and then reused when the <J> expectation values have 
# to be evaluated. the routine uses the Icalc_parstorage matrix for storing the 
# rotated matrices.
# One should experiment with the value of X and compare it with using the full matrix 
# diagonalisation before relying on the results. 
# This approximation method is probably not necessary for d-electron matrices (as 
# those are usually quite small), but for f-electrons it is quite useful!
# Uncomment the next line to use the truncate method with 1/5th of all levels

# truncate_matrix=0.2

# a list of crystal field parameters (Llm denote Wybourne parameters)
units=meV
L22S=-4.8939
L21S=-3.26255
L20=2.8304
L21=-2.44691
L22=1.42739
L44S=0.0574029
L43S=-0.0425222
L42S=0.00309947
L41S=0.0862038
L40=0.0162131
L41=0.0646528
L42=-0.000904011
L43=0.11307
L44=0.0900338
L66S=0.00305141
L65S=0.006398
L64S=0.00110317
L63S=0.00151215
L62S=0.00344795
L61S=-0.00449046
L60=-0.00389599
L61=-0.00336784
L62=-0.00100565
L63=-0.00402095
L64=0.00173027
L65=0.000486626
L66=-0.00348322

# Blm denote Stevens normalised parameters. They are related to Wyborne
# parameters by: Blm = l(lm) t(lm) Llm  where l(lm) is a l- and m-dependent
# factor and t(lm) is the Stevens operator equivalent factor, usually 
# denoted alpha, beta, gamma for l=2,4,6 respectively. Please note that
# by default the factor <J||alpha||J> etc. for a J-multiplet is used for
# f-electron ions whilst <L||alpha||L> etc. ignoring spin is used for 
# d-electron ions as per convention in the literature. You can force the
# use of one or the other type of factors by uncommenting the following:

#use_L_operator_equivalent
#use_J_operator_equivalent

B20=-0.001

# Note that the parameters output in the header of the ic1ion.out file
# is given in the type of the last input parameter read (e.g. Stevens
# normalisation in this case).

# magnetisation calculation can be triggered by the following parameters
calcmag
# default magnetisation output units is Bohr magneton per ion

# if the following is uncommented, ic1ion will output magnetisation in emu/mol
#emu

# if the following is uncommented, ic1ion will output magnetisation in Am^2/mol
#simag

 xT   = 1
 xHa  = 0
 xHb  = 0
 xHc  = 0
 xmin = 1
 xstep= 1
 xmax = 300
 yT   = 0
 yHa  = 1
 yHb  = 0
 yHc  = 0
 ymin = 1
 ystep= 1
 ymax = 1

#-------------------------------------------------------------------------------------------
# the following parameters are read, when ic1ion is used as a module for other
# programs in the mcphase program suite, such as mcphas, mcdiff, mcdisp ...

#-------------------------------------------------------
# Neutron Scattering Length (10^-12 cm) (can be complex)
#-------------------------------------------------------
SCATTERINGLENGTHREAL=0.249
SCATTERINGLENGTHIMAG=0
#  ... note: - if an occupancy other than 1.0 is needed, just reduce 
#              the scattering length linear accordingly
 
 
#--------------------------------------------------------------------------------------
# Neutron Magnetic Form Factor coefficients - thanks to J Brown
#   d = 2*pi/Q      
#   s = 1/2/d = Q/4/pi   
#   sin(theta) = lambda * s
#   r= s*s = Q*Q/16/pi/pi
#
#   <j0(Qr)>=   FFj0A*EXP(-FFj0a*r) + FFj0B*EXP(-FFj0b*r) + FFj0C*EXP(-FFj0c*r) + FFj0D
#   <j2(Qr)>=r*(FFj2A*EXP(-FFj2a*r) + FFj2B*EXP(-FFj2b*r) + FFj2C*EXP(-FFj2c*r) + FFj2D
#   <j4(Qr)>=r*(FFj4A*EXP(-FFj4a*r) + FFj4B*EXP(-FFj4b*r) + FFj4C*EXP(-FFj4c*r) + FFj4D
#   <j6(Qr)>=r*(FFj6A*EXP(-FFj6a*r) + FFj6B*EXP(-FFj6b*r) + FFj6C*EXP(-FFj6c*r) + FFj6D
#
#   Dipole Approximation for Neutron Magnetic Formfactor:
#        -Spin Form Factor       FS(Q)=<j0(Q)>
#        -Angular Form Factor    FL(Q)=<j0(Q)>+<j2(Q)>
#        -Rare Earth Form Factor F(Q) =<j0(Q)>+<j2(Q)>*(2/gJ-1)
#--------------------------------------------------------------------------------------
FFj0A=0.4085 FFj0a=23.8526 FFj0B=0.6091 FFj0b=8.2456 FFj0C=-0.1676 FFj0c=0.0415 FFj0D=0.1496     
FFj2A=3.4386 FFj2a=16.5303 FFj2B=1.9638 FFj2b=6.1415 FFj2C=0.2997 FFj2c=2.2669 FFj2D=0.0009     
FFj4A=-0.4759 FFj4a=14.0462 FFj4B=0.2747 FFj4b=3.7306 FFj4C=0.2458 FFj4c=1.2504 FFj4D=0.0057 


#---------------------------------------------------------
# radial wave function parameters (for use in programs pointc, chrgplt, charges) 
# R_Np,XIp(r)= r^(Np-1) . exp(-xi r) . (2 XIp)^(Np+0.5) / sqrt(2Np!)  
# values tabulated in clementi & roetti Atomic data and nuclear data tables 14 (1974) 177-478
# Co2+ is isoelectronic to Fe+, looking at page  422 of Clemente & Roetti 
# the 3D radial wave function is expanded as R(r)=sum_p C_p R_Np,XIp(r)
#---------------------------------------------------------
N1=3 XI1=4.95296 C1=0.36301 
N2=3 XI2=12.2963 C2=0.02707 
N3=3 XI3=7.03565 C3=0.14777
N4=3 XI4=2.74850 C4=0.49771 
N5=3 XI5=1.69027 C5=0.11388 
# for program pointcharge alternatively, the radial matrix elements may be given:

#---------------------------------------------------------
# Radial Matrix Elements (e.g. Abragam Bleaney 1971 p 399)
#---------------------------------------------------------
#<r^2> in units of a0^2 a0=0.5292 Angstroem
R2=1.24653
#<r^4> in units of a0^4 a0=0.5292 Angstroem
R4=3.67281
#<r^6> in units of a0^6 a0=0.5292 Angstroem
R6=21.0652

#---------------------------------------------------------
#pointcharges charge[|e|]  x[A] y[A] z[A]
#---------------------------------------------------------
pointcharge=  0.3            3    4    2


#-------------------------------------------------------
# Debye-Waller Factor: sqr(Intensity)~|sf|~EXP(-2 * DWF *s*s)=EXP (-W)
#                      with s=sin(theta)/lambda=Q/4pi
# relation to other notations: 2*DWF=Biso=8 pi^2 <u^2>
# unit of DWF is [A^2]
#-------------------------------------------------------
DWF=0.4
\end{verbatim}
}
The output files are written to directory {\prg results} and contain energy eigentstates ({\prg ic1ion.out}) ,
polycrystal neutron transitions and intensities ({\prg ic1ion.trs}) and, optional, %%@
the magnetisation ({\prg ic1ion.mag}). Note in these
output files the parameters alpha/beta/gamma
in the header are not the Steven's operator equivalent factors, but
rather the parameters of the configuration interactions. The operator equivalent factors are listed as 
{\tt <L||alpha||L>} etc. or {\tt <J||alpha||J>} etc. depending on whether $J$- or $L$-operator equivalents
are used. $L$- ($J$-)operators are default for $d$- ($f$-)electron ions. In practice, as the $J$- or $L$- 
operators are valid only within an $L$- or $J$-manifold, whilst {\prg ic1ion} calculates the splitting of the
full $l^\nu$ configuration with many manifolds, the operator equivalent factors only serve to convert the input
Stevens parameters into Wybourne normalisation, which is used internally by the program to calculate the
Hamiltonian, for details on crystal field parameter conventions see appendix~\ref{cfparconventions}.


Note that if you wish to compare the output of {\prg ic1ion} with published energies/wavefunctions using Stevens
parameters for rare earths, you should set the Slater {\prg F2, F4, F6} and spin-orbit {\prg zeta} integrals to
very high values - such as 10$^7$ meV, in order to obtain the $LS$-coupling limit. For comparison with Stevens
parameters within an $L$-manifold for $d$-electron ions, the Slater integrals should be set to high
values but the spin-orbit integral to zero. In this case the {\prg ic1ion} output will have twice the
degeneracy of a pure $L$-manifold calculation because the spin is taken into account in {\prg ic1ion} and
generates an additional two-fold degeneracy. If a non-zero (e.g. literature specified) spin-orbit integral
{\prg zeta} is specified, then this degeneracy will be lifted.

\subsection{{\prg cpic1ion} - calculation of the specific heat from {\prg ic1ion.out}}
 The crystal field contribution to the specific heat may be calculated 
from the output file {\prg ic1ion.out} using the  program {\prg cpic1ion}, e.g. {\bf cpic1ion 10 100 1 [options]}
calculates the specific heat in the temperature interval 10-100 K with a step width
of 1 K. Alternatively a comparison to experimental data can be made by {\bf cpic1ion 1 2 cpexp.dat},
where the temperatures are given in column 1 and the experimental specific heat in column
2 of file cpexp.dat. The calculated specific heat is compared to the experimental data and
a standard deviation {\em sta} is calculated and output is written to stdout.
Other quantities can be calculated using the options: -s  (calculate entropy  (J/molK) instead of cp),
-f (calculate free energy (J/mol) instead of cp),-u  (calculate magnetic energy (J/mol) instead of cp),
-z (calculate partition sum instead of cp).


\subsection{using {\prg ic1ion\index{ic1ion}} as a module in {\prg mcphas},{\prg mcdiff},{\prg mcdisp},...}

Currently {\prg ic1ion\index{ic1ion}} is loadable by mcphas\index{mcphas} as an external module. The single
ion parameter file is just the same as given in the above format for running {\prg ic1ion\index{ic1ion}} as 
a stand alone program. However, some input files have to follow special conventions in order that 
{\prg mcphas, mcdisp, mcdiff} work correctly: 

{\bf Note:} in case of intermediate coupling
it is necessary to specify
L and S seperately (in interactions and Zeeman term and neutron cross sections ...)
therefore  ''gJ=0'' in {\prg mcphas.j\index{mcphas.j}} has a special meaning:
\begin{itemize}
\item  it modifies
the convention that Ja=Jy=O1-1(J) Jb=Jz=O10(J) Jc=Jx=O11(J) and instead
puts Ja=Sx Jb=Lx Jc=Sy Jd=Ly Je=Sz Jf=Lz, the Jn for $n>f$ being defined in the single
ion module. 
\item for higher order operators, the current McPhase definitions are expanded, so instead of
Jd==O22S, we made Jg==T22S, Jh==T21S, Ji==T20, Jj==T21, Jk==T22,
Jl==T33S, etc... the operators $\hat T_{kq}$ are described in equation~(\ref{tkqhermitian})
% these operators are to be understood as in the Stevens notation, equivalent
%operators to tesseral harmonic functions stripped by their prefactors $c_{lm}$ (factors before 
%the
%bracket in table IV of~\cite{hutchings64-227} and in appendix~\ref{tesseral}), 
%i.e. $\langle v| O_{lm}| v' \rangle=\langle nu |\sum_i Z_l^m(\Omega_i)|nu' \rangle / c_{lm}$.
\item the programs {\prg mcphas, mcdisp etc} take into account
 a modified Zeeman term $H_{Ze}=-\mu_B(2\mbf S+\mbf L)\mbf H$.
 However, in program {\prg singleion\index{singleion}} the meanfields of S and L can be
 separately given by specifying gjmbH=$g \mu_B H$ with g=2 or 1 depending
 on whether it is spin or orbital component of the field. 
\item the magnetic moments are calculated and shown in plots according 
to $\mbf M=\mu_B(2<\mbf S>+<\mbf L>)$
\item Also the inelastic neutron scattering cross section in {\prg mcdisp\index{mcdisp}} is evaluated using the first
6x6 components of the dynamical matrix  in dipole approximation for the magnetic formfactor.
\end{itemize}

\subsection{{\prg icf1ion} - Intermediate Crystal Field module}

In addition to the full {\prg ic1ion} module, there is a derived module, {\prg icf1ion}, in which only the lowest
energy term of constant $L$ and $S$ is considered. Thus, the Coulomb interaction (see next section) is supposed
to have such high energy that the low temperature physical properties may be accounted for just by the spin-orbit,
crystal field, and exchange interactions. As a smaller energy matrix is required in this module, calculations will
be faster. In the cases of a single outer shell electron or hole (e.g. Ce$^{3+}$ or Yb$^{3+}$), the two modules are 
identical. {\prg icf1ion} accepts the same inputs as {\prg ic1ion}, except that the Slater integrals $F^k$ are 
ignored. 

Currently (version 4.0), the beyond dipole part of {\prg mcdisp} is not yet implemented, as are the calculations
of the spin and orbital moment densities. All other packages may use {\prg icf1ion}.

%\vspace{1cm}
\subsection{Formalism}

As noted in equation~\ref{ichamilton}, in the case of intermediate coupling, we calculate
the single-ion Hamiltonian considering the electron-electron (Coulomb) interaction and the
spin-orbit (SO), which are respectively the first\footnote{Note that we do not calculate all the
terms in the braces in the first line of equation~\ref{ichamilton}. In particular using the
\emph{central field approximation} the first summation produces Hydrogen-like energy levels, called
\emph{configurations}, which are split into \emph{terms} by the second summation. What we refer to
as the Coulomb interaction is \emph{only} this second summation, and we shall consider the lowest
energy configuration only. This configuration corresponds to the outer most electrons in ion, and is
labelled $nl^\nu$, for example, $4f^1$ for Ce$^{3+}$.} and second lines of the equation, in addition
to the crystal field (CF) interaction which {\prg cfield} treats. 

The term \emph{intermediate coupling} is somewhat a misnomer, because it does not refer to a
specific coupling scheme to determine the total angular momentum $J$, such as
\emph{Russel-Saunders} coupling ($LS$-coupling), or $jj$-coupling. Rather, it refers the case where all
three of the above single-ion interactions are treated on an equal footing. This is to distinguish
between the practice of using only the lowest energy manifold of levels with the same $J$ in either
of the other two coupling schemes to calculate the low temperature (low energy) physical
properties\footnote{For the case of $LS$-coupling this corresponds to ignoring the Coulomb and SO
interactions, which are in this limit both much larger than the CF. For $jj$-coupling, the Coulomb
interaction is treated as small and neglected, but the spin-orbit interaction is considered.}.

Finaly in some cases, \emph{intermediate coupling}, also refers to the narrower case where although
the Coulomb and SO terms are included, their strengths are fixed by parameters determined from
experimental (optical) spectra, or ab-initio (Hartree-Fock) calculations, rather than being freely
varied.  

In the case where the Coulomb interaction dominates, \emph{and} the
spin-orbit interaction is sufficiently stronger than the CF, then only the lowest
energy \emph{multiplet} - that is, the degenerate energy levels produced by the Coulomb and SO
interactions - need to be taken into account. This multiplet may be determined from \emph{Hund's
Rules}, and hence we need only calculate the CF interaction (as \emph{cfield} does). This case is
termed \emph{Russel-Saunders} or $LS$-coupling. 
Thus  \emph{Russel-Saunders} coupling corresponds to the limit where
both the Coulomb and spin-orbit interaction are large (but where $\mathcal{H}_{\mathrm{Coulomb}}\gg
\mathcal{H}_{\mathrm{SO}}\gg
\mathcal{H}_{\mathrm{CEF}}$), i.e. the crystal field interaction is treated as a perturbation. Only
the lowest energy spin-orbit multiplet is considered, and its degeneracy is split by the CF. These
levels are labeled by the total angular momentum quantum numbers $L, S, J$ and $J_z$\footnote{Also
denoted $m_J$.}. This is the case treated in the modules {\prg so1ion}/{\prg cfield}, and is 
sometimes called the \emph{weak field} case.

The other limit corresponds to $jj$-coupling, where the spin-orbit interaction
dominates $\mathcal{H}_{\mathrm{SO}}\gg
\mathcal{H}_{\mathrm{Coulomb}}$, and there is no term or multiplet structure, 
in this case only the single electron total
angular momentum quantum number $j$ serves to label the states. 
In between these two limits is thus, \emph{intermediate coupling}. 

In addition to the limits represented by these coupling schemes, the crystal field literature
also introduces several other limiting cases, depending on the strength of the crystal
field. The \emph{weak field} limit corresponds to $LS$-coupling. The \emph{intermediate field} is
the case where $\mathcal{H}_{\mathrm{Coulomb}}\gg\mathcal{H}_{\mathrm{CEF}}\gg
\mathcal{H}_{\mathrm{SO}}$ and this case may be calculated using the {\prg icf1ion} module,
which ignores $\mathcal{H}_{\mathrm{Coulomb}}$, and treats only $\mathcal{H}_{\mathrm{CEF}}$ and
$\mathcal{H}_{\mathrm{SO}}$ within the lowest term $^{2S+1} L$ determined by Hund's rules.

The \emph{strong field} limit is where $\mathcal{H}_{\mathrm{CF}}$ dominates even over the Coulomb
interaction. In this case, the crystal field is considered to act first on single-electron states,
splitting the degenerate orbital levels $m_l=-l,...,l$, as dictated by the point symmetry of the
ion (e.g. for a transition metal ion in cubic symmetry this is usually into a doublet
$|m_l=0,2\rangle$, labelled by the group theoretical irreducible representation of the cubic group,
$E_g$; and a triplet $|m_l=-2,-1,1\rangle$ labelled $T_{2g}$). Electrons may then fill these single
electron crystal field states, and their degeneracy may be lifted by the Coulomb and spin-orbit
interactions. This case is not explicitly considered in the module, but one may calculate the
resulting energy levels by increasing the crystal field above the Coulomb interaction. {\prg
ic1ion} however, uses the total orbital and spin angular momenta as labels for the states, rather
than single-electron angular momenta or irreducible representations, so it is difficult to intepret
{\prg ic1ion} output in terms of this limit.

Finally, {\prg ic1ion} may be used in the general case, where all the three main
single ion interactions are considered with variable strengths
$\mathcal{H}_{\mathrm{Coulomb}}\sim
\mathcal{H}_{\mathrm{SO}}\sim
\mathcal{H}_{\mathrm{CEF}}$, but may also specifically refer to
the situation in which the strengths of the Coulomb and spin-orbit interaction is fixed by
parameters determined either from experimental (optical) spectra or ab-initio (Hartree-Fock)
calculations. If the CF interaction is relatively weak, there will still be a multiplet structure
but the eigenstates of the system is a mixture of the $LS$-basis states (due to the SO interaction),
with only $J$ being a good quantum number which can be used to distinguish the states. In the case
of a large CF interaction, however, \emph{J-mixing} can occur, which results in the lost of a
distinct multiplet structure.

The operators corresponding to each of the three single-ion interactions are calculated as matrices
using $LS$-coupling basis states, which are labelled according to a group theoretical
classification. This is to facilitate the use of \emph{coefficients of fractional parentage} (cfp)
in the calculation of the matrix elements. These coefficients allow states of given configuration
$nl^\nu$ to be determined by the states of a parent configuration $nl^{\nu-1}$ with one fewer electron.
That is, we express the $\nu^{\mathrm{th}}$ electron state $|\Omega\rangle$ as a sum,
$\sum_{\bar{\Omega},\omega} \langle\bar{\Omega};\omega|\Omega\rangle|\omega\rangle$, over all possible ways of
adding a single electron $|\omega\rangle$ to a the $(\nu-1)^{\mathrm{th}}$ state, $|\bar{\Omega}\rangle$,
where $\langle\bar{\Omega};\omega|\Omega\rangle$ is the cfp. We also use the results that a single-particle
operator, $\langle\Omega|V|\Omega'\rangle$, acting on $\nu$ electrons is just $\nu$ times the same operator 
acting on one electron, $\langle\omega|v|\omega'\rangle$, to expand the states (and matrix element) in
terms of the cfp's,

\[ \langle\Omega|V|\Omega'\rangle = \nu \sum_{\bar{\Omega},\omega,\omega'} \langle\Omega|\bar{\Omega};\omega\rangle
                                         \langle\omega|v|\omega'\rangle \langle\bar{\Omega};\omega'|\Omega'\rangle \]

\noindent Once the cfp are known, only the single electron matrix element $\langle\omega|v|\omega'\rangle$ 
is needed to be determined for each particular operator. This is generally a straigtforward task. The
cfp are also dependent on the classification of the states for which they are constructed, but the
advantage of the use of a classification based on group theory is that we can use Racah's
theorem~\cite{racah49-1352}\footnote{Equation 11}, to factorise the cfp into terms involving only
pairs of quantum numbers labelling the states, rather than all the quantum numbers together. Because
of the properties of the groups from which they derive, most of these factors do not have explicit
algebraic formulae\footnote{Except for the case of the rotation group in 3 dimensions, SO(3) (whose
representations are labelled by the angular momentum quantum numbers, $L$ and $J$), where they are
simply the $3j$ symbols}. However they make a chain calculation of the cfp from a single electron
state practicable. These calculations were carried out in the atomic spectroscopy community in the
1950's and '60's and tabulated in the works of Racah~\cite{racah49-1352}, Wybourne~\cite{wybourne61}, 
Allison and McNulty~\cite{allison74}\footnote{http://cpc.cs.qub.ac.uk/cpc/cgi-bin/showversions.pl/?catid=acry},
and Nielson and Koster~\cite{nielson63-1}.

The LS-coupling basis states are tabulated by Nielsen and Koster~\cite{nielson63-1}, and the
formulae for the matrix elements are from Racah~\cite{racah42-186,racah42-438,racah43-367}
 and Judd~\cite{judd88-1}.

%We construct the cfp's from several different factors. This comes about
%because you can factorise the cfp into terms involving any two quantum
%numbers of the set you used to label your basis states, if the
%quantum numbers are labels for a representation of a group (hence the
%use of group theory to determine the basis states). In this case the
%factors involving the quantum numbers S,L,J,mJ are just 3j-symbols.
%The factors involving v, U, and n are not, and for these I've used
%tables constructed by Racah, Wybourne and later from a program written
%by Allison~\cite{allison74}.

%Some checks of the cfp's were made - by comparing
%them to values tabulated by Cowan, and by using the reciprocity
%relation between the coefficients like $l^{n-1}a"L"S"|\}l^{n}aLS$ and
%$l^{n-1}aLS|\}l^{n}a"L"S"$. These seem to check out... but I was mostly
%concentrating on the configurations with fewer electrons - e.g. f1-f5.
%Finally I don't use these tables for d-electrons at all, because
%there are not so many states involved, and all the cfp's have been
%tabulated by Nielson and Koster~\cite{nielson63-1}, so
%It is just programmed them in as a look up table. 

Finally expressions based on the above equation for the matrix elements of the spin-orbit and
crystal field operators were derived by Elliot, Judd and Runciman~\cite{elliot57-509}\footnote{Eqn
16 for SO, and 25-27 for CF, and reproduced in more modern notation in appendix C
of~\cite{ducle09-1}}. For the Coulomb interaction, we have chosen to use the operators defined by
Racah~\cite{racah49-1352}\footnote{Eqns 66 for the conversion of the Racah parameters to Slater
integrals, 63 for the $\hat e_0$ operator, 69 for $\hat e_1$, 73-74 for $\hat e_2$, and 78, 28 and 80-87 for
$\hat e_3$}. In the case of the rank-1 spin and orbital angular momentum operators, an explicit algebraic
expression exists for the sum over cfp in the above equation~\cite{judd88-1}\footnote{Eqn 7-58},
resulting in the equations quoted by Chan and Lam~\cite{chan70-219}.

The interested reader will find much more details in the books by Judd~\cite{judd88-1} and
Wybourne~\cite{wybourne65}, and for the more mathematically inclined, in the lecture notes of
Racah~\cite{racah51}. In addition, this group theory approach was used to determine the neutron
cross-sections in a series of papers in the 1970's and 1980's by Balcar and Lovesey, and summarised
in their books~\cite{lovesey84-1,balcar89-1}.

Finally, the Slater and spin-orbit integrals which parameterise the strength of the Coulomb and SO
interactions (these are the radial integrals, whereas what we have calculated using the above
techniques is effectively the angular parts of these operators), are tabulated in various books. In
particular we have used the experimentally determined values given by Abragam
Bleaney~\cite{abragam70-1}, Griffith~\cite{griffith71-1} and Chakravarty~\cite{chakravarty80-1} for
the transition metal ions, and by Carnall et al.~\cite{carnall89,carnall92} for the rare earths.
Other relevant literature are the famous papers by Racah~\cite{racah42-186,racah42-438,racah43-367}.

%The first (electrostatic) term is well known to be exactly solvable only for Hydrogen. Thus we use
%the well known \emph{central field approximation} to separate it into a Hydrogen-like part which
%produces degenerate energy levels called \emph{configurations} labeled by the principal quantum
%number, $n$, and the orbital quantum number $l$ of the electrons; and a remainder term which
%resembles the second summation in the first line of equation~\ref{ichamilton}, and lifts the
%degeneracy of the configurations, producing energy levels called \emph{terms} which are labeled by
%the total spin, $S$, and orbital, $L$, angular momentum of the multi-electron system. 
%
%In order to make the calculations tractable we shall consider only the configuration with the lowest
%energy, which in general belongs to the outer most electron shell of the ion of interest. For
%Lanthanide ions, these electrons have $n=4$ and $l=3$, and are thus labeled $4f$, whilst actinide
%ions are $5f$, and the 1st, 2nd, and 3rd transition metal serires (with $l=3$) are $3d$, $4d$ and
%$5d$ respectively. 
%
%Futhermore the terms are split by the spin-orbit interation into \emph{multiplets} labeled by the
%total orbital angular momentum, $J=L+S$, which is finally split by the crystal field. \emph{cfield}
%considers only the lowest lying multiplet and the crystal field splitting, which is usually adequate
%for rare earth systems because the Coulomb and SO terms are large enough such that the energy gap
%between multiplet is too large to affect the low (or room) temperature physical properties of such
%systems.

%In order to calculate the matrix elements of each of these interactions, we refer to the formulae in
%the following:

We now list formulae for the matrix elements of the principal operators used in {\prg ic1ion} as
they are programmed in:
\begin{equation}
 \mathcal H = \mathcal H_{\mathrm{coulomb}} + \mathcal H_{\mathrm {spinorbit}} + \mathcal H_{\mathrm{crystalfield}} + \mathcal H_{\mathrm{zeman}}
\end{equation}

\begin{quotation}
\item[\bf Coulomb electron electron interaction operator:]
\[ \mathcal{H}_{\mathrm{coulomb}} = \sum_{i>j=1}^\nu \frac{e^2}{4\pi\epsilon_0|\mathbf{r}_i-\mathbf{r}_j|} = 
    \sum_{k} F^k \sum_{ij} {\mathbf T}_i^{(k)} \cdot {\mathbf T}_j^{(k)} = \sum_k F^k \hat f_k \]

\noindent The denominator $|\mathbf{r}_i-\mathbf{r}_j|$ was expanded as a multipole series in terms 
of spherical harmonic functions $Y_{kq}$, which are in turn related to the tensor operators ${\mathbf
T}_i^{(k)}$ of the $i^{\mathrm th}$ electron. The radial dependence of the electron wavefunction is
then embodied in the \emph{Slater integrals}, $F^k$, whilst the angular part is given by the Slater
operators $\hat f_k$, and $k$ runs through even integers from 0 to $2l$. In the case of $f$-electrons 
we use Racah's operators, $\hat e_k$ which are linear combinations of $\hat f_k$ which transform according to
specific irreducible representations of the group SO(7)$\times G_2$. This means that their matrix
elements are more straightforward to calculate. The relation between these operators and $\hat f_k$, and
their matrix elements are given by Racah~\cite{racah49-1352}. For $d$- and $p$-electrons the $\hat f_k$
operators are used directly, from tabulated matrix elements given by~\cite{nielson63-1}. 
%e_0 E^0 + e_1 E^1 + e_2 E^2 + e_3 E^3 \]
%The denominator $|\mathbf{r}_i-\mathbf{r}_j|$ in the Coulomb interaction may be expanded as in a multipole
%series in terms of spherical harmonics, whilst the CF operators are themselves tensor operators which transform
%like the spherical harmonic functions. Thus we can (after some algebra) express the Coulomb interaction as
%$\sum_{ij} e^2 \sum_k (r_<^k/r_>^k) \mathbf{T}^{(k)}_i \cdot \mathbf{T}^{(k)}_j$, where $r_<$ is the lesser of
%$r_i$ or $r_j$ and $r_>$ is the greater; and $\mathbf{T}^{(k)}$ is a vector of operators $T^{(k)}_q$, with
%$q=-k,\ldots,k$.
%
%After some further manipulations the following expression may be derived:
%
%
%Eqns 66 (conversion from Racah parameters, $E^k$ to Slater integrals, $F_k$), 63 ($\hat e_0$ operator), 
%69 ($e1$), eqns 73-74 ($e2$), and eqns 78,28,80-87 ($e3$), from the important paper by Giulio
%Racah~\cite{racah49-1352}.
\\
\item[\bf Spin-orbit interaction operator $\mathcal H_{\mathrm {spinorbit}}=\zeta \sum_i ({\mbf s}_i . {\mbf l}_i)$:]
Eqn 16 from Elliot, Judd and Runciman~\cite{elliot57-509}.
\\


\item[\bf Crystal field interaction:]

The crystal field interaction is given by the electrical potential produced according 
to Coulombs law by neighbouring charges $q_j$ at positions $\mbf R_j$ acting on
n electrons.
 
\begin{equation}\label{crystal field}
\mathcal H_{\mathrm{crystalfield}}=\frac{1}{4\pi \epsilon_0}\sum_{i=1}^\nu \sum_j \frac{-|e|q_j}{|\mbf R_j - \mbf r_i|}
\end{equation}

The crystal field is rewritten in terms of tesseral harmonics $Z_{kq}(\Omega)$ (see appendix \ref{tesseral})

\begin{equation}
\mathcal H_{\mathrm{crystalfield}}=-|e|\sum_{i=1}^\nu \sum_j q_j \sum_{k=0}^{\infty}\frac{r_i^k}{\epsilon_0 R_j^{k+1}} \sum_{q=-k}^k \frac{1}{2k+1} %%@
Z_{kq}(\Omega_i)Z_{kq}(\Omega_j)
\end{equation}

we define crystal field parameters 

\begin{equation}
\gamma_{kq}=\sum_j \frac{q_j}{2k+1}\frac{1}{\epsilon_0R_j^{k+1}}Z_{kq}(\Omega_j)
\end{equation}

and rewrite the crystal field as

\begin{equation}
\mathcal H_{\mathrm{crystalfield}}=-|e| \sum_{i=1}^\nu \sum_{k=0}^{\infty} \sum_{q=-k}^k r_i^k \gamma_{kq}Z_{kq}(\Omega_i)
\end{equation}

Defining rescaled tesseral harmonics $z_{kq}$ as

\begin{eqnarray}
z_{kq}=\sqrt{\frac{8\pi}{2k+1}}Z_{kq} \dots q\neq 0 \\
z_{k0}=\sqrt{\frac{4\pi}{2k+1}}Z_{k0} \dots q=0 \\
\end{eqnarray}

and Wybourne parameters as

\begin{eqnarray}
L_{kq}=-|e|\langle r^k \rangle \gamma_{kq}\sqrt{\frac{2k+1}{8\pi}} \dots q\neq 0 \\
L_{k0}=-|e|\langle r^k \rangle \gamma_{kq}\sqrt{\frac{2k+1}{4\pi}} \dots q= 0 \\
\end{eqnarray}

the crystal field may be rewritten as

\begin{equation}
\mathcal H_{\mathrm{crystalfield}}=  \sum_{k=0}^{\inf} \sum_{q=-k}^k  L_{kq} \sum_{i=1}^\nu z_{kq}(\Omega_i)
\end{equation}

In order to calculate the crystal field Hamiltonian matrix, we replace the tesseral harmonic
functions $\sum_i z_{kq}(\Omega_i)$ by the tensor operators $\hat{T}_{kq}$ which transform in the
same way as the tesseral harmonics under the action of the rotation group SO(3). That is, the action
of the operator $\hat{T}_{kq}$ on a wavefunction $|J,m_J\rangle$ is the same as that of the
functions $z_{kq}$ on the coordinates $\Omega \equiv (\theta,\phi)$. The $\hat{T}_{kq}$ are
expressed in terms of similar operators $\hat{C}_{kq}$ (which transform in the same way as spherical
harmonics) as:

\begin{equation} \label{tkqhermitian}
  \hat{T}_{k0} = \hat{C}_{k0}, \qquad \hat{T}_{k,\pm|q|} = \sqrt{\pm 1} \left[
  \hat{C}_{k,-|q|} \pm (-1)^{|q|} \hat{C}_{k,|q|} \right]
\end{equation}

\noindent and the matrix elements of these operators are given by 

\begin{equation} \label{iccfmatel}
\begin{array}{l}
\langle \theta J m_J | \hat{C}_{kq} | \theta' J' m_J' \rangle = \delta_{SS'} (-1)^{J-m_J}
       \left( \begin{array}{ccc} J & k & J' \\ -m_J & q & m'_J \end{array} \right)
       \quad \langle l | \hat{c}_k | l \rangle  \\
 \quad \times \nu \sum_{\bar{\theta}} (\theta\{|\bar{\theta}) (\theta'\{|\bar{\theta})
       \ \  (-1)^{\bar{L}+L+k+l} ([L][L'])^{\frac{1}{2}}
       \left\{ \begin{array}{ccc} L' & k & L \\ l & \bar{L} & l \end{array} \right\}  \\
 \quad \times \qquad\qquad\qquad\qquad\  (-1)^{S+L'+k+J} ([J][J'])^{\frac{1}{2}}
       \left\{ \begin{array}{ccc} J' & k & J \\ L & S & L' \end{array} \right\}
\end{array}
\end{equation}

\noindent with the single electron matrix element

\begin{equation} \label{iccf1ematel}
\langle l | \hat{c}_k | l \rangle = (-1)^l (2l+1) \left( \begin{array}{ccc} l & k & l \\ 0 & 0 & 0
\end{array} \right)
\end{equation}

%This may be expressed as ....
%
%{\bf some formula for CEF as it is programmed in ???? note that we have changed
%racah\_ukq by a factor $(-1)^q$ and also the operators (Ukqp $\pm Ukqm$) had their sign changed
%for negative q !!! relation to the $T^{(k)}_q$ ???}
%
%The main formulae for $T^{(k)}_q$ is in chapter 7 of the Judd book (eqn. 7-51 and
%7-52)~\cite{judd88-1}. Eqns 25-27 from Elliot, Judd and Runciman~\cite{elliot57-509}, 
%and in slightly more modern notation
%in equation C.14 from the thesis of Duc Le~\cite{ducle09-1}. 
%Note also the equivalence between the 3j symbols and Clebsch-Gordan coefficients
%(eqn 19 here: http://mathworld.wolfram.com/Wigner3j-Symbol.html ), and
%between the 6j symbols and Racah's W coefficient (http://mathworld.wolfram.com/RacahW-Coefficient.html ).
%The basic matrix element is the same for all operators written
%variously as Ukq, Vkq and Tkq except for the single electron reduced
%matrix element $<nl||?||n'l'>$. The unit tensor operator Ukq have
%$<lm||uk||l'm'>=\delta_{nn'}\delta_{ll'}$ whilst the Vkq operators have
%$<lm||vk||l'm'>=\sqrt{2k+1}\delta_{nn'}\delta_{ll'}$
%and the spherical
%harmonics tensor operator Tkq, has
%\begin{equation}
%<lm||vk||l'm'>=(-1)^l*\sqrt{(2l+1)(2l'+1)}*\left (\begin{tabular}{ccc}  l & k & l' \\ 0 & 0 & 0 \\ \end{tabular}\right)
%\end{equation}
%
\item[\bf The magnetic moment operators, $L_x L_y L_z S_x S_y S_z$ and $\mathcal H_{\mathrm{zeman}}$ :]

The orbital and spin operators have the same dependence on $m_J$, due to the Wigner-Eckart theorem:

\begin{eqnarray}
\bra{\theta Jm_J}(L,S)_{x,y}\ket{\theta'J'm_J'} &=& \frac{(-1)^{J-m_J}}{\sqrt{\pm 2}}
 \left[ \threej{J&1&J'}{-m_J& 1&m'_J} \pm \threej{J&-1&J'}{-m_J&1&m'_J} \right] \\ \nonumber  
 && \qquad \times \bra{\theta J}|(L,S)||\ket{\theta'J'} \\
    \bra{\theta Jm_J}(L,S)_z\ket{\theta'J'm_J'} &=& (-1)^{J-m_J}
    \threej{J&1&J'}{-m_J&0&m'_J} \bra{\theta J}|(L,S)|\ket{\theta'J'}
\end{eqnarray}

\noindent where the reduced matrix elements are:

\begin{eqnarray} \label{eq:muredmat} \nonumber
&&\bra{\theta J}|(L,S)|\ket{\theta'J'} = \delta_{\theta,\theta'} (-1)^{S+L+(J,J')} \sqrt{(2J+1)(2J'+1)} \\
&&\;\;\times \sqrt{(L,S)((L,S)+1)(2(L,S)+1)} \sixj{J'&1&J}{ (L,S) &
(S,L) & (L,S)'} 
\end{eqnarray}

Using this the Zeman interaction can be written as
\begin{equation}
 \mathcal H_{\mathrm{zeman}}= -\mu_B (2\mbf S^n+\mbf L^n) {\mbf H}
\end{equation}


%Defined in Chan and Lam~\cite{chan70-219} and again in modern notation in equation C.19 of~\cite{ducle09-1}.
\end{quotation}

\subsection{formalism for going beyond the dipolar approximation}\label{ic1ionmq}

{\prg ic1ion} is prepared to do calculations of the neutron cross section going beyond the dipole approximation for 
the scattering cross section. The formulation of the scattering operator
 implemented in the functions {\prg mq} and {\prg dncalc} have been given by Lovesey and Balcar~\cite{lovesey84-1}, 6.87b:
 
 \begin{eqnarray}
 \langle  l^\nu v S L J M| \hat \mathcal Q_q | l^\nu v' S' L' J' M' \rangle &=& \sqrt{4\pi}\sum_{K',Q,Q'} \left [ %%@
Y_{K'-1}^{Q}(\hat \mbf Q) \left (\frac{2K'+1}{K'+1}\right ) \right. \\
&& \times \{ A(K'-1,K')+B(K'-1,K')(K'-1 Q K' Q'|1q)\}\\
&& \left . + Y_{K'}^{Q}(\hat \mbf Q) B(K',K') (K' Q K' Q'|1q) \right ] (K' Q' J' M'|JM)
 \end{eqnarray}
 
... these are the spherical components with $q=-1,0,1$, which are related to the cartesian components by

\begin{eqnarray}
\hat \mathcal Q_x &=& +\frac{1}{\sqrt{2}}(\hat \mathcal Q_{+1} +\hat \mathcal Q_{-1})\\
\hat \mathcal Q_y &=& -\frac{i}{\sqrt{2}}(\hat \mathcal Q_{+1} -\hat \mathcal Q_{-1}) \\
\hat \mathcal Q_z &=& \hat \mathcal Q_0  \\
\end{eqnarray}

The coefficients $A(K,K')$ and $B(K,K')$ are complex to obtain, formulas will not be given here, we refer the reader %%@
to~\cite{lovesey84-1} and just mention, that the computation involves
of 3j, 6j and 9j symbols and some fractional parentage coefficients. 
